%%%%%%%%%%%%%%%%%%%%%%%%%%%%%%%%%%%%%%%%%%%%%%%%%%%%%%%%%%%%%%%%%%%%%%%%
%   LaTeX source code to approximate a NIST Technical report
%	Instructions for authors: tinyurl.com/techpubsnist 
%	DOI watermark will be added on final PDF
% 	Developed by K. Miller, kmm5@nist.gov 
%	Last updated: 22-January-2021
%%%%%%%%%%%%%%%%%%%%%%%%%%%%%%%%%%%%%%%%%%%%%%%%%%%%%%%%%%%%%%%%%%%
\documentclass[12pt]{article}
\usepackage{amsmath}
\usepackage{amsfonts}   % if you want the fonts
\usepackage{amssymb}    % if you want extra symbols
\usepackage{graphicx}   % need for figures
\usepackage{xcolor}
\usepackage{caption}
\usepackage{bm}
\usepackage{secdot}	
\usepackage{listings}	
\usepackage{mathptmx}
\usepackage{float}
\usepackage[utf8]{inputenc}
\usepackage{textcomp}
\usepackage[hang,flushmargin,bottom]{footmisc} % footnote format
\setlength{\parindent}{0pt}

% colori per il codice

\definecolor{codegreen}{rgb}{0,0.6,0}
\definecolor{codegray}{rgb}{0.5,0.5,0.5}
\definecolor{codepurple}{rgb}{0.58,0,0.82}
\definecolor{backcolour}{rgb}{0.95,0.95,0.92}

\lstdefinestyle{mystyle}{
    backgroundcolor=\color{backcolour},   
    commentstyle=\color{codegreen},
    keywordstyle=\color{magenta},
    numberstyle=\tiny\color{codegray},
    stringstyle=\color{codepurple},
    basicstyle=\ttfamily\footnotesize,
    breakatwhitespace=false,         
    breaklines=true,                 
    captionpos=b,                    
    keepspaces=true,                 
    numbers=left,                    
    numbersep=5pt,                  
    showspaces=false,                
    showstringspaces=false,
    showtabs=false,                  
    tabsize=2
}

\lstset{style=mystyle}

\usepackage{titlesec}
\titleformat{\section}{\normalsize\bfseries}{\thesection.}{1em}{}	% required for heading numbering style
\titleformat*{\subsection}{\normalsize\bfseries}

\usepackage{tocloft}	% change typeset, titles, and format list of appendices/figures/tables
\renewcommand{\cftdot}{}	
\renewcommand{\contentsname}{Table of Contents}
\renewcommand{\cftpartleader}{\cftdotfill{\cftdotsep}} % for parts
\renewcommand{\cftsecleader}{\cftdotfill{\cftdotsep}}
\renewcommand\cftbeforesecskip{\setlength{4pt}{}}
\addtolength{\cftfignumwidth}{1em}
\renewcommand{\cftfigpresnum}{\figurename\ }
\addtolength{\cfttabnumwidth}{1em}
\renewcommand{\cfttabpresnum}{\tablename\ }
\setlength{\cfttabindent}{0in}    %% adjust as you like
\setlength{\cftfigindent}{0in} 

\usepackage{enumitem}         % to control spacing between bullets/numbered lists

% \usepackage[numbers,sort&compress]{natbib} % format bibliography 
% \renewcommand{\bibsection}{}
% \setlength{\bibsep}{20.0pt}

\usepackage[hidelinks]{hyperref}
\hypersetup{
	colorlinks = true,
urlcolor ={blue},
citecolor = {.},
linkcolor = {.},
anchorcolor = {.},
filecolor = {.},
menucolor = {.},
runcolor = {.}
pdftitle={},%%put title here to auto-fill properties of the PDF
pdfsubject={},%%put abstract here
pdfauthor={}, %%put author list here
pdfkeywords={} %%put keywords here
}
\urlstyle{same}

\usepackage{epstopdf} % converting EPS figure files to PDF

\usepackage{fancyhdr, lastpage}	% formatting document, calculating number of pages, formatting headers
\setlength{\topmargin}{-0.5in}
\setlength{\headheight}{39pt}
\setlength{\oddsidemargin}{0.25in}
\setlength{\evensidemargin}{0.25in}
\setlength{\textwidth}{6.0in}
\setlength{\textheight}{8.5in}

\usepackage{caption} % required for Figure labels
\captionsetup{font=small,labelfont=bf,figurename=Fig.,labelsep=period,justification=raggedright} 

%%%%%%%%%%% !!!!!! REQUIRED - FILL OUT METADATA HERE !!!!!!!! %%%%%%%%%%%%%%
%  	Report Number - fill in Report Number sent to you (see info below)
%   DOI Statement - fill in DOI sent to you 
%   Month Year - fill in Month and Year of Publication
%%%%%%%%%%%%%%%%%%%%%%%%%%%%%%%%%%%%%%%%%%%%%%%%%%%%%%%%%%%%%%%%%%%%%%%%%%%%%%%%%%%%%%
\newcommand{\pubnumber}{1500-XX}
\newcommand{\DOI}{https://doi.org/10.6028/NIST.SP.1500-XX}
\newcommand{\monthyear}{Month Year}
%%%%%%%%%%%%%%%%%%%%%%%%%%%%%%%%%%%%%%%%%%%%%%%%%%%%%%%%%%%%%%%%%%%%
%   	BEGIN DOCUMENT 
%%%%%%%%%%%%%%%%%%%%%%%%%%%%%%%%%%%%%%%%%%%%%%%%%%%%%%%%%%%%%%%%%%%%
\begin{document}
\urlstyle{rm} % Format style of \url   

%%%%%%%%%%%%%%%%%%%%%%%%%%%%%%%%%%%%%%%%%%%%%%%%%%%%%%%%%%%%%%%%%%%%
%   Cover Page is REQUIRED and must contain the information 
%	displayed here, at a minimum. Additional artwork may be included 
%	(e.g., official project/conference logo, etc.).
%	Pub Number automated based on metadata
%%%%%%%%%%%%%%%%%%%%%%%%%%%%%%%%%%%%%%%%%%%%%%%%%%%%%%%%%%%%%%%%%%%%
\begin{titlepage}
    \begin{flushright}
        %%%%%%%%%%%%%%%%%%%%%%%%%%%%%%%%%%%%%%%%%%%%%%%%%%%%%%%%%%%%%%%%%%%%
        % 	Automated based on metadata - delete if not applicable
        %%%%%%%%%%%%%%%%%%%%%%%%%%%%%%%%%%%%%%%%%%%%%%%%%%%%%%%%%%%%%%%%%%%%
        \LARGE{\textbf{Molecular Simulations Report}}\\
        \vfill
        %%%%%%%%%%%%%%%%%%%%%%%%%%%%%%%%%%%%%%%%%%%%%%%%%%%%%%%%%%%%%%%%%%%%
        %	Title 
        %%%%%%%%%%%%%%%%%%%%%%%%%%%%%%%%%%%%%%%%%%%%%%%%%%%%%%%%%%%%%%%%%%%%
        \Huge{\textbf{Comparative analysis of water structure for
                complex 1fk9}}\\
        \vfill
        %%%%%%%%%%%%%%%%%%%%%%%%%%%%%%%%%%%%%%%%%%%%%%%%%%%%%%%%%%%%%%%%%%%%
        %	Authors - add complete list of authors, affiliations will be 
        %   added on title page
        %%%%%%%%%%%%%%%%%%%%%%%%%%%%%%%%%%%%%%%%%%%%%%%%%%%%%%%%%%%%%%%%%%%%
        \large Conforto Filippo - 2021856\\

        \vfill


        \includegraphics[width=0.3\linewidth]{logo.png}\\


    \end{flushright}
\end{titlepage}

%%%%%%%%%%%%%%%%%%%%%%%%%%%%%%%%%%%%%%%%%%%%%%%%%%%%%%%%%%%%%%%%%%%%
%   Table of Contents is required
% 	List of Tables & Figures required if more than 5 tables/figures
%%%%%%%%%%%%%%%%%%%%%%%%%%%%%%%%%%%%%%%%%%%%%%%%%%%%%%%%%%%%%%%%%%%%
\begin{center}
    \tableofcontents
    \listoffigures
\end{center}
\pagebreak
\section{Introduction}
The following report contains procedures and results regarding different tests of pre-equilibrated water solvents for the molecular complex 1fk9\cite{Ren2000}.

1fk9 represents the combination of two chains, both coming from
the HIV-1 reverse transcriptase\cite{pmid3040055}.
The first chain (chain A) in particular contains residues from position
588 to 1130, while the second one contains residues from
position 588 to 1027. One additional remark to do on the first
chain is that it contains a modified residue of cystine (CYS)
called CSD. Along with this particular monomer another
important molecule contained in the complex is the EFZ ligand
that will need its specific modelization before being able to
do any simulation on the complex behaviour.

The CSD residue can be found in correspondence to the residue
867 of HIV reverse transcriptase original sequence, and can be
found in the middle of the Chain A, while the EFZ residue works
as a ligand between the two chains.

By considering such structure the main goal of the work done
was to compare how the molecular complex behaves while
constrained in different solutions. To do so different
model preparation and molecular dynamics tools were used to
prepare and simulate the system.
\section{Methods}
The main applicative used to get the final results is Amber \cite{Amber}, in particular the tools tleap, pdb4amber and sander.

The complex structure was retrieved from the pdb dataset, and contained all the already named chains and residues. The file contained also some HOH elements (Water codification) that were useless for the project goal.

Since the two residues (CSD and EFZ) were not associated to any force field, the parameters were manually added through model preparation.

Steps followed were the same for both the residues, involving these procedures:

\begin{itemize}
    \item Hydrogen addition
    \item Charges computation
    \item General force field (gaff) assignment
\end{itemize}

After these procedures, the final parameters were saved in a .lib file ready to be loaded for the final simulation process. 

The molecular complex was then loaded in tleap and minimized using sander, after loading the necessary force fields and the manually created ones. 

The next step was so to add a preequilibrated solute, in our system. 


\bibliographystyle{plain} % We choose the "plain" reference style
\bibliography{bibl} % Entries are in the refs.bib file

\end{document}
