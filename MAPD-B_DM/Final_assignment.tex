\documentclass[11pt]{article}

    \usepackage[breakable]{tcolorbox}
    \usepackage{parskip} % Stop auto-indenting (to mimic markdown behaviour)
    
    \usepackage{iftex}
    \ifPDFTeX
    	\usepackage[T1]{fontenc}
    	\usepackage{mathpazo}
    \else
    	\usepackage{fontspec}
    \fi

    % Basic figure setup, for now with no caption control since it's done
    % automatically by Pandoc (which extracts ![](path) syntax from Markdown).
    \usepackage{graphicx}
    % Maintain compatibility with old templates. Remove in nbconvert 6.0
    \let\Oldincludegraphics\includegraphics
    % Ensure that by default, figures have no caption (until we provide a
    % proper Figure object with a Caption API and a way to capture that
    % in the conversion process - todo).
    \usepackage{caption}
    \DeclareCaptionFormat{nocaption}{}
    \captionsetup{format=nocaption,aboveskip=0pt,belowskip=0pt}

    \usepackage{float}
    \floatplacement{figure}{H} % forces figures to be placed at the correct location
    \usepackage{xcolor} % Allow colors to be defined
    \usepackage{enumerate} % Needed for markdown enumerations to work
    \usepackage{geometry} % Used to adjust the document margins
    \usepackage{amsmath} % Equations
    \usepackage{amssymb} % Equations
    \usepackage{textcomp} % defines textquotesingle
    % Hack from http://tex.stackexchange.com/a/47451/13684:
    \AtBeginDocument{%
        \def\PYZsq{\textquotesingle}% Upright quotes in Pygmentized code
    }
    \usepackage{upquote} % Upright quotes for verbatim code
    \usepackage{eurosym} % defines \euro
    \usepackage[mathletters]{ucs} % Extended unicode (utf-8) support
    \usepackage{fancyvrb} % verbatim replacement that allows latex
    \usepackage{grffile} % extends the file name processing of package graphics 
                         % to support a larger range
    \makeatletter % fix for old versions of grffile with XeLaTeX
    \@ifpackagelater{grffile}{2019/11/01}
    {
      % Do nothing on new versions
    }
    {
      \def\Gread@@xetex#1{%
        \IfFileExists{"\Gin@base".bb}%
        {\Gread@eps{\Gin@base.bb}}%
        {\Gread@@xetex@aux#1}%
      }
    }
    \makeatother
    \usepackage[Export]{adjustbox} % Used to constrain images to a maximum size
    \adjustboxset{max size={0.9\linewidth}{0.9\paperheight}}

    % The hyperref package gives us a pdf with properly built
    % internal navigation ('pdf bookmarks' for the table of contents,
    % internal cross-reference links, web links for URLs, etc.)
    \usepackage{hyperref}
    % The default LaTeX title has an obnoxious amount of whitespace. By default,
    % titling removes some of it. It also provides customization options.
    \usepackage{titling}
    \usepackage{longtable} % longtable support required by pandoc >1.10
    \usepackage{booktabs}  % table support for pandoc > 1.12.2
    \usepackage[inline]{enumitem} % IRkernel/repr support (it uses the enumerate* environment)
    \usepackage[normalem]{ulem} % ulem is needed to support strikethroughs (\sout)
                                % normalem makes italics be italics, not underlines
    \usepackage{mathrsfs}
    

    
    % Colors for the hyperref package
    \definecolor{urlcolor}{rgb}{0,.145,.698}
    \definecolor{linkcolor}{rgb}{.71,0.21,0.01}
    \definecolor{citecolor}{rgb}{.12,.54,.11}

    % ANSI colors
    \definecolor{ansi-black}{HTML}{3E424D}
    \definecolor{ansi-black-intense}{HTML}{282C36}
    \definecolor{ansi-red}{HTML}{E75C58}
    \definecolor{ansi-red-intense}{HTML}{B22B31}
    \definecolor{ansi-green}{HTML}{00A250}
    \definecolor{ansi-green-intense}{HTML}{007427}
    \definecolor{ansi-yellow}{HTML}{DDB62B}
    \definecolor{ansi-yellow-intense}{HTML}{B27D12}
    \definecolor{ansi-blue}{HTML}{208FFB}
    \definecolor{ansi-blue-intense}{HTML}{0065CA}
    \definecolor{ansi-magenta}{HTML}{D160C4}
    \definecolor{ansi-magenta-intense}{HTML}{A03196}
    \definecolor{ansi-cyan}{HTML}{60C6C8}
    \definecolor{ansi-cyan-intense}{HTML}{258F8F}
    \definecolor{ansi-white}{HTML}{C5C1B4}
    \definecolor{ansi-white-intense}{HTML}{A1A6B2}
    \definecolor{ansi-default-inverse-fg}{HTML}{FFFFFF}
    \definecolor{ansi-default-inverse-bg}{HTML}{000000}

    % common color for the border for error outputs.
    \definecolor{outerrorbackground}{HTML}{FFDFDF}

    % commands and environments needed by pandoc snippets
    % extracted from the output of `pandoc -s`
    \providecommand{\tightlist}{%
      \setlength{\itemsep}{0pt}\setlength{\parskip}{0pt}}
    \DefineVerbatimEnvironment{Highlighting}{Verbatim}{commandchars=\\\{\}}
    % Add ',fontsize=\small' for more characters per line
    \newenvironment{Shaded}{}{}
    \newcommand{\KeywordTok}[1]{\textcolor[rgb]{0.00,0.44,0.13}{\textbf{{#1}}}}
    \newcommand{\DataTypeTok}[1]{\textcolor[rgb]{0.56,0.13,0.00}{{#1}}}
    \newcommand{\DecValTok}[1]{\textcolor[rgb]{0.25,0.63,0.44}{{#1}}}
    \newcommand{\BaseNTok}[1]{\textcolor[rgb]{0.25,0.63,0.44}{{#1}}}
    \newcommand{\FloatTok}[1]{\textcolor[rgb]{0.25,0.63,0.44}{{#1}}}
    \newcommand{\CharTok}[1]{\textcolor[rgb]{0.25,0.44,0.63}{{#1}}}
    \newcommand{\StringTok}[1]{\textcolor[rgb]{0.25,0.44,0.63}{{#1}}}
    \newcommand{\CommentTok}[1]{\textcolor[rgb]{0.38,0.63,0.69}{\textit{{#1}}}}
    \newcommand{\OtherTok}[1]{\textcolor[rgb]{0.00,0.44,0.13}{{#1}}}
    \newcommand{\AlertTok}[1]{\textcolor[rgb]{1.00,0.00,0.00}{\textbf{{#1}}}}
    \newcommand{\FunctionTok}[1]{\textcolor[rgb]{0.02,0.16,0.49}{{#1}}}
    \newcommand{\RegionMarkerTok}[1]{{#1}}
    \newcommand{\ErrorTok}[1]{\textcolor[rgb]{1.00,0.00,0.00}{\textbf{{#1}}}}
    \newcommand{\NormalTok}[1]{{#1}}
    
    % Additional commands for more recent versions of Pandoc
    \newcommand{\ConstantTok}[1]{\textcolor[rgb]{0.53,0.00,0.00}{{#1}}}
    \newcommand{\SpecialCharTok}[1]{\textcolor[rgb]{0.25,0.44,0.63}{{#1}}}
    \newcommand{\VerbatimStringTok}[1]{\textcolor[rgb]{0.25,0.44,0.63}{{#1}}}
    \newcommand{\SpecialStringTok}[1]{\textcolor[rgb]{0.73,0.40,0.53}{{#1}}}
    \newcommand{\ImportTok}[1]{{#1}}
    \newcommand{\DocumentationTok}[1]{\textcolor[rgb]{0.73,0.13,0.13}{\textit{{#1}}}}
    \newcommand{\AnnotationTok}[1]{\textcolor[rgb]{0.38,0.63,0.69}{\textbf{\textit{{#1}}}}}
    \newcommand{\CommentVarTok}[1]{\textcolor[rgb]{0.38,0.63,0.69}{\textbf{\textit{{#1}}}}}
    \newcommand{\VariableTok}[1]{\textcolor[rgb]{0.10,0.09,0.49}{{#1}}}
    \newcommand{\ControlFlowTok}[1]{\textcolor[rgb]{0.00,0.44,0.13}{\textbf{{#1}}}}
    \newcommand{\OperatorTok}[1]{\textcolor[rgb]{0.40,0.40,0.40}{{#1}}}
    \newcommand{\BuiltInTok}[1]{{#1}}
    \newcommand{\ExtensionTok}[1]{{#1}}
    \newcommand{\PreprocessorTok}[1]{\textcolor[rgb]{0.74,0.48,0.00}{{#1}}}
    \newcommand{\AttributeTok}[1]{\textcolor[rgb]{0.49,0.56,0.16}{{#1}}}
    \newcommand{\InformationTok}[1]{\textcolor[rgb]{0.38,0.63,0.69}{\textbf{\textit{{#1}}}}}
    \newcommand{\WarningTok}[1]{\textcolor[rgb]{0.38,0.63,0.69}{\textbf{\textit{{#1}}}}}
    
    
    % Define a nice break command that doesn't care if a line doesn't already
    % exist.
    \def\br{\hspace*{\fill} \\* }
    % Math Jax compatibility definitions
    \def\gt{>}
    \def\lt{<}
    \let\Oldtex\TeX
    \let\Oldlatex\LaTeX
    \renewcommand{\TeX}{\textrm{\Oldtex}}
    \renewcommand{\LaTeX}{\textrm{\Oldlatex}}
    % Document parameters
    % Document title
    \title{MAPDB - Final Assignment}
    
    
    
    
    
% Pygments definitions
\makeatletter
\def\PY@reset{\let\PY@it=\relax \let\PY@bf=\relax%
    \let\PY@ul=\relax \let\PY@tc=\relax%
    \let\PY@bc=\relax \let\PY@ff=\relax}
\def\PY@tok#1{\csname PY@tok@#1\endcsname}
\def\PY@toks#1+{\ifx\relax#1\empty\else%
    \PY@tok{#1}\expandafter\PY@toks\fi}
\def\PY@do#1{\PY@bc{\PY@tc{\PY@ul{%
    \PY@it{\PY@bf{\PY@ff{#1}}}}}}}
\def\PY#1#2{\PY@reset\PY@toks#1+\relax+\PY@do{#2}}

\@namedef{PY@tok@w}{\def\PY@tc##1{\textcolor[rgb]{0.73,0.73,0.73}{##1}}}
\@namedef{PY@tok@c}{\let\PY@it=\textit\def\PY@tc##1{\textcolor[rgb]{0.25,0.50,0.50}{##1}}}
\@namedef{PY@tok@cp}{\def\PY@tc##1{\textcolor[rgb]{0.74,0.48,0.00}{##1}}}
\@namedef{PY@tok@k}{\let\PY@bf=\textbf\def\PY@tc##1{\textcolor[rgb]{0.00,0.50,0.00}{##1}}}
\@namedef{PY@tok@kp}{\def\PY@tc##1{\textcolor[rgb]{0.00,0.50,0.00}{##1}}}
\@namedef{PY@tok@kt}{\def\PY@tc##1{\textcolor[rgb]{0.69,0.00,0.25}{##1}}}
\@namedef{PY@tok@o}{\def\PY@tc##1{\textcolor[rgb]{0.40,0.40,0.40}{##1}}}
\@namedef{PY@tok@ow}{\let\PY@bf=\textbf\def\PY@tc##1{\textcolor[rgb]{0.67,0.13,1.00}{##1}}}
\@namedef{PY@tok@nb}{\def\PY@tc##1{\textcolor[rgb]{0.00,0.50,0.00}{##1}}}
\@namedef{PY@tok@nf}{\def\PY@tc##1{\textcolor[rgb]{0.00,0.00,1.00}{##1}}}
\@namedef{PY@tok@nc}{\let\PY@bf=\textbf\def\PY@tc##1{\textcolor[rgb]{0.00,0.00,1.00}{##1}}}
\@namedef{PY@tok@nn}{\let\PY@bf=\textbf\def\PY@tc##1{\textcolor[rgb]{0.00,0.00,1.00}{##1}}}
\@namedef{PY@tok@ne}{\let\PY@bf=\textbf\def\PY@tc##1{\textcolor[rgb]{0.82,0.25,0.23}{##1}}}
\@namedef{PY@tok@nv}{\def\PY@tc##1{\textcolor[rgb]{0.10,0.09,0.49}{##1}}}
\@namedef{PY@tok@no}{\def\PY@tc##1{\textcolor[rgb]{0.53,0.00,0.00}{##1}}}
\@namedef{PY@tok@nl}{\def\PY@tc##1{\textcolor[rgb]{0.63,0.63,0.00}{##1}}}
\@namedef{PY@tok@ni}{\let\PY@bf=\textbf\def\PY@tc##1{\textcolor[rgb]{0.60,0.60,0.60}{##1}}}
\@namedef{PY@tok@na}{\def\PY@tc##1{\textcolor[rgb]{0.49,0.56,0.16}{##1}}}
\@namedef{PY@tok@nt}{\let\PY@bf=\textbf\def\PY@tc##1{\textcolor[rgb]{0.00,0.50,0.00}{##1}}}
\@namedef{PY@tok@nd}{\def\PY@tc##1{\textcolor[rgb]{0.67,0.13,1.00}{##1}}}
\@namedef{PY@tok@s}{\def\PY@tc##1{\textcolor[rgb]{0.73,0.13,0.13}{##1}}}
\@namedef{PY@tok@sd}{\let\PY@it=\textit\def\PY@tc##1{\textcolor[rgb]{0.73,0.13,0.13}{##1}}}
\@namedef{PY@tok@si}{\let\PY@bf=\textbf\def\PY@tc##1{\textcolor[rgb]{0.73,0.40,0.53}{##1}}}
\@namedef{PY@tok@se}{\let\PY@bf=\textbf\def\PY@tc##1{\textcolor[rgb]{0.73,0.40,0.13}{##1}}}
\@namedef{PY@tok@sr}{\def\PY@tc##1{\textcolor[rgb]{0.73,0.40,0.53}{##1}}}
\@namedef{PY@tok@ss}{\def\PY@tc##1{\textcolor[rgb]{0.10,0.09,0.49}{##1}}}
\@namedef{PY@tok@sx}{\def\PY@tc##1{\textcolor[rgb]{0.00,0.50,0.00}{##1}}}
\@namedef{PY@tok@m}{\def\PY@tc##1{\textcolor[rgb]{0.40,0.40,0.40}{##1}}}
\@namedef{PY@tok@gh}{\let\PY@bf=\textbf\def\PY@tc##1{\textcolor[rgb]{0.00,0.00,0.50}{##1}}}
\@namedef{PY@tok@gu}{\let\PY@bf=\textbf\def\PY@tc##1{\textcolor[rgb]{0.50,0.00,0.50}{##1}}}
\@namedef{PY@tok@gd}{\def\PY@tc##1{\textcolor[rgb]{0.63,0.00,0.00}{##1}}}
\@namedef{PY@tok@gi}{\def\PY@tc##1{\textcolor[rgb]{0.00,0.63,0.00}{##1}}}
\@namedef{PY@tok@gr}{\def\PY@tc##1{\textcolor[rgb]{1.00,0.00,0.00}{##1}}}
\@namedef{PY@tok@ge}{\let\PY@it=\textit}
\@namedef{PY@tok@gs}{\let\PY@bf=\textbf}
\@namedef{PY@tok@gp}{\let\PY@bf=\textbf\def\PY@tc##1{\textcolor[rgb]{0.00,0.00,0.50}{##1}}}
\@namedef{PY@tok@go}{\def\PY@tc##1{\textcolor[rgb]{0.53,0.53,0.53}{##1}}}
\@namedef{PY@tok@gt}{\def\PY@tc##1{\textcolor[rgb]{0.00,0.27,0.87}{##1}}}
\@namedef{PY@tok@err}{\def\PY@bc##1{{\setlength{\fboxsep}{\string -\fboxrule}\fcolorbox[rgb]{1.00,0.00,0.00}{1,1,1}{\strut ##1}}}}
\@namedef{PY@tok@kc}{\let\PY@bf=\textbf\def\PY@tc##1{\textcolor[rgb]{0.00,0.50,0.00}{##1}}}
\@namedef{PY@tok@kd}{\let\PY@bf=\textbf\def\PY@tc##1{\textcolor[rgb]{0.00,0.50,0.00}{##1}}}
\@namedef{PY@tok@kn}{\let\PY@bf=\textbf\def\PY@tc##1{\textcolor[rgb]{0.00,0.50,0.00}{##1}}}
\@namedef{PY@tok@kr}{\let\PY@bf=\textbf\def\PY@tc##1{\textcolor[rgb]{0.00,0.50,0.00}{##1}}}
\@namedef{PY@tok@bp}{\def\PY@tc##1{\textcolor[rgb]{0.00,0.50,0.00}{##1}}}
\@namedef{PY@tok@fm}{\def\PY@tc##1{\textcolor[rgb]{0.00,0.00,1.00}{##1}}}
\@namedef{PY@tok@vc}{\def\PY@tc##1{\textcolor[rgb]{0.10,0.09,0.49}{##1}}}
\@namedef{PY@tok@vg}{\def\PY@tc##1{\textcolor[rgb]{0.10,0.09,0.49}{##1}}}
\@namedef{PY@tok@vi}{\def\PY@tc##1{\textcolor[rgb]{0.10,0.09,0.49}{##1}}}
\@namedef{PY@tok@vm}{\def\PY@tc##1{\textcolor[rgb]{0.10,0.09,0.49}{##1}}}
\@namedef{PY@tok@sa}{\def\PY@tc##1{\textcolor[rgb]{0.73,0.13,0.13}{##1}}}
\@namedef{PY@tok@sb}{\def\PY@tc##1{\textcolor[rgb]{0.73,0.13,0.13}{##1}}}
\@namedef{PY@tok@sc}{\def\PY@tc##1{\textcolor[rgb]{0.73,0.13,0.13}{##1}}}
\@namedef{PY@tok@dl}{\def\PY@tc##1{\textcolor[rgb]{0.73,0.13,0.13}{##1}}}
\@namedef{PY@tok@s2}{\def\PY@tc##1{\textcolor[rgb]{0.73,0.13,0.13}{##1}}}
\@namedef{PY@tok@sh}{\def\PY@tc##1{\textcolor[rgb]{0.73,0.13,0.13}{##1}}}
\@namedef{PY@tok@s1}{\def\PY@tc##1{\textcolor[rgb]{0.73,0.13,0.13}{##1}}}
\@namedef{PY@tok@mb}{\def\PY@tc##1{\textcolor[rgb]{0.40,0.40,0.40}{##1}}}
\@namedef{PY@tok@mf}{\def\PY@tc##1{\textcolor[rgb]{0.40,0.40,0.40}{##1}}}
\@namedef{PY@tok@mh}{\def\PY@tc##1{\textcolor[rgb]{0.40,0.40,0.40}{##1}}}
\@namedef{PY@tok@mi}{\def\PY@tc##1{\textcolor[rgb]{0.40,0.40,0.40}{##1}}}
\@namedef{PY@tok@il}{\def\PY@tc##1{\textcolor[rgb]{0.40,0.40,0.40}{##1}}}
\@namedef{PY@tok@mo}{\def\PY@tc##1{\textcolor[rgb]{0.40,0.40,0.40}{##1}}}
\@namedef{PY@tok@ch}{\let\PY@it=\textit\def\PY@tc##1{\textcolor[rgb]{0.25,0.50,0.50}{##1}}}
\@namedef{PY@tok@cm}{\let\PY@it=\textit\def\PY@tc##1{\textcolor[rgb]{0.25,0.50,0.50}{##1}}}
\@namedef{PY@tok@cpf}{\let\PY@it=\textit\def\PY@tc##1{\textcolor[rgb]{0.25,0.50,0.50}{##1}}}
\@namedef{PY@tok@c1}{\let\PY@it=\textit\def\PY@tc##1{\textcolor[rgb]{0.25,0.50,0.50}{##1}}}
\@namedef{PY@tok@cs}{\let\PY@it=\textit\def\PY@tc##1{\textcolor[rgb]{0.25,0.50,0.50}{##1}}}

\def\PYZbs{\char`\\}
\def\PYZus{\char`\_}
\def\PYZob{\char`\{}
\def\PYZcb{\char`\}}
\def\PYZca{\char`\^}
\def\PYZam{\char`\&}
\def\PYZlt{\char`\<}
\def\PYZgt{\char`\>}
\def\PYZsh{\char`\#}
\def\PYZpc{\char`\%}
\def\PYZdl{\char`\$}
\def\PYZhy{\char`\-}
\def\PYZsq{\char`\'}
\def\PYZdq{\char`\"}
\def\PYZti{\char`\~}
% for compatibility with earlier versions
\def\PYZat{@}
\def\PYZlb{[}
\def\PYZrb{]}
\makeatother


    % For linebreaks inside Verbatim environment from package fancyvrb. 
    \makeatletter
        \newbox\Wrappedcontinuationbox 
        \newbox\Wrappedvisiblespacebox 
        \newcommand*\Wrappedvisiblespace {\textcolor{red}{\textvisiblespace}} 
        \newcommand*\Wrappedcontinuationsymbol {\textcolor{red}{\llap{\tiny$\m@th\hookrightarrow$}}} 
        \newcommand*\Wrappedcontinuationindent {3ex } 
        \newcommand*\Wrappedafterbreak {\kern\Wrappedcontinuationindent\copy\Wrappedcontinuationbox} 
        % Take advantage of the already applied Pygments mark-up to insert 
        % potential linebreaks for TeX processing. 
        %        {, <, #, %, $, ' and ": go to next line. 
        %        _, }, ^, &, >, - and ~: stay at end of broken line. 
        % Use of \textquotesingle for straight quote. 
        \newcommand*\Wrappedbreaksatspecials {% 
            \def\PYGZus{\discretionary{\char`\_}{\Wrappedafterbreak}{\char`\_}}% 
            \def\PYGZob{\discretionary{}{\Wrappedafterbreak\char`\{}{\char`\{}}% 
            \def\PYGZcb{\discretionary{\char`\}}{\Wrappedafterbreak}{\char`\}}}% 
            \def\PYGZca{\discretionary{\char`\^}{\Wrappedafterbreak}{\char`\^}}% 
            \def\PYGZam{\discretionary{\char`\&}{\Wrappedafterbreak}{\char`\&}}% 
            \def\PYGZlt{\discretionary{}{\Wrappedafterbreak\char`\<}{\char`\<}}% 
            \def\PYGZgt{\discretionary{\char`\>}{\Wrappedafterbreak}{\char`\>}}% 
            \def\PYGZsh{\discretionary{}{\Wrappedafterbreak\char`\#}{\char`\#}}% 
            \def\PYGZpc{\discretionary{}{\Wrappedafterbreak\char`\%}{\char`\%}}% 
            \def\PYGZdl{\discretionary{}{\Wrappedafterbreak\char`\$}{\char`\$}}% 
            \def\PYGZhy{\discretionary{\char`\-}{\Wrappedafterbreak}{\char`\-}}% 
            \def\PYGZsq{\discretionary{}{\Wrappedafterbreak\textquotesingle}{\textquotesingle}}% 
            \def\PYGZdq{\discretionary{}{\Wrappedafterbreak\char`\"}{\char`\"}}% 
            \def\PYGZti{\discretionary{\char`\~}{\Wrappedafterbreak}{\char`\~}}% 
        } 
        % Some characters . , ; ? ! / are not pygmentized. 
        % This macro makes them "active" and they will insert potential linebreaks 
        \newcommand*\Wrappedbreaksatpunct {% 
            \lccode`\~`\.\lowercase{\def~}{\discretionary{\hbox{\char`\.}}{\Wrappedafterbreak}{\hbox{\char`\.}}}% 
            \lccode`\~`\,\lowercase{\def~}{\discretionary{\hbox{\char`\,}}{\Wrappedafterbreak}{\hbox{\char`\,}}}% 
            \lccode`\~`\;\lowercase{\def~}{\discretionary{\hbox{\char`\;}}{\Wrappedafterbreak}{\hbox{\char`\;}}}% 
            \lccode`\~`\:\lowercase{\def~}{\discretionary{\hbox{\char`\:}}{\Wrappedafterbreak}{\hbox{\char`\:}}}% 
            \lccode`\~`\?\lowercase{\def~}{\discretionary{\hbox{\char`\?}}{\Wrappedafterbreak}{\hbox{\char`\?}}}% 
            \lccode`\~`\!\lowercase{\def~}{\discretionary{\hbox{\char`\!}}{\Wrappedafterbreak}{\hbox{\char`\!}}}% 
            \lccode`\~`\/\lowercase{\def~}{\discretionary{\hbox{\char`\/}}{\Wrappedafterbreak}{\hbox{\char`\/}}}% 
            \catcode`\.\active
            \catcode`\,\active 
            \catcode`\;\active
            \catcode`\:\active
            \catcode`\?\active
            \catcode`\!\active
            \catcode`\/\active 
            \lccode`\~`\~ 	
        }
    \makeatother

    \let\OriginalVerbatim=\Verbatim
    \makeatletter
    \renewcommand{\Verbatim}[1][1]{%
        %\parskip\z@skip
        \sbox\Wrappedcontinuationbox {\Wrappedcontinuationsymbol}%
        \sbox\Wrappedvisiblespacebox {\FV@SetupFont\Wrappedvisiblespace}%
        \def\FancyVerbFormatLine ##1{\hsize\linewidth
            \vtop{\raggedright\hyphenpenalty\z@\exhyphenpenalty\z@
                \doublehyphendemerits\z@\finalhyphendemerits\z@
                \strut ##1\strut}%
        }%
        % If the linebreak is at a space, the latter will be displayed as visible
        % space at end of first line, and a continuation symbol starts next line.
        % Stretch/shrink are however usually zero for typewriter font.
        \def\FV@Space {%
            \nobreak\hskip\z@ plus\fontdimen3\font minus\fontdimen4\font
            \discretionary{\copy\Wrappedvisiblespacebox}{\Wrappedafterbreak}
            {\kern\fontdimen2\font}%
        }%
        
        % Allow breaks at special characters using \PYG... macros.
        \Wrappedbreaksatspecials
        % Breaks at punctuation characters . , ; ? ! and / need catcode=\active 	
        \OriginalVerbatim[#1,codes*=\Wrappedbreaksatpunct]%
    }
    \makeatother

    % Exact colors from NB
    \definecolor{incolor}{HTML}{303F9F}
    \definecolor{outcolor}{HTML}{D84315}
    \definecolor{cellborder}{HTML}{CFCFCF}
    \definecolor{cellbackground}{HTML}{F7F7F7}
    
    % prompt
    \makeatletter
    \newcommand{\boxspacing}{\kern\kvtcb@left@rule\kern\kvtcb@boxsep}
    \makeatother
    \newcommand{\prompt}[4]{
        {\ttfamily\llap{{\color{#2}[#3]:\hspace{3pt}#4}}\vspace{-\baselineskip}}
    }
    

    
    % Prevent overflowing lines due to hard-to-break entities
    \sloppy 
    % Setup hyperref package
    \hypersetup{
      breaklinks=true,  % so long urls are correctly broken across lines
      colorlinks=true,
      urlcolor=urlcolor,
      linkcolor=linkcolor,
      citecolor=citecolor,
      }
    % Slightly bigger margins than the latex defaults
    
    \geometry{verbose,tmargin=1in,bmargin=1in,lmargin=1in,rmargin=1in}
    
    

\begin{document}

\maketitle




\hypertarget{filippo-conforto}{%
    \subsection*{Filippo Conforto}\label{filippo-conforto}}

\hypertarget{id-2021856}{%
    \subsection*{ID: 2021856}\label{id-2021856}}

\hypertarget{libraries-and-imports}{%
    \subsubsection*{Libraries and imports}\label{libraries-and-imports}}

\begin{tcolorbox}[breakable, size=fbox, boxrule=1pt, pad at break*=1mm,colback=cellbackground, colframe=cellborder]
    \prompt{In}{incolor}{2}{\boxspacing}
    \begin{Verbatim}[commandchars=\\\{\}]
        \PY{k+kn}{import} \PY{n+nn}{numpy} \PY{k}{as} \PY{n+nn}{np}
        \PY{k+kn}{import} \PY{n+nn}{time}
        \PY{k+kn}{import} \PY{n+nn}{requests} \PY{k}{as} \PY{n+nn}{req}
        \PY{k+kn}{import} \PY{n+nn}{json}
        \PY{k+kn}{from} \PY{n+nn}{functools} \PY{k+kn}{import} \PY{n}{reduce}
        \PY{k+kn}{import} \PY{n+nn}{matplotlib}\PY{n+nn}{.}\PY{n+nn}{pyplot} \PY{k}{as} \PY{n+nn}{plt}
        \PY{k+kn}{import} \PY{n+nn}{matplotlib}\PY{n+nn}{.}\PY{n+nn}{axis} \PY{k}{as} \PY{n+nn}{mplax}
    \end{Verbatim}
\end{tcolorbox}

\hypertarget{fun-exercise}{%
    \subsection*{Fun Exercise}\label{fun-exercise}}

The \texttt{XOR} operation between the given numbers returns the correct
results. The proof for each given couple follows:

\begin{itemize}
    \tightlist
    \item
          1 \(\oplus\) 2 = 01 \(\oplus\) 10 = 11 = 3
    \item
          2 \(\oplus\) 5 = 010 \(\oplus\) 101 = 111 = 7
    \item
          3 \(\oplus\) 7 = 011 \(\oplus\) 111 = 100 = 4
    \item
          4 \(\oplus\) 5 = 100 \(\oplus\) 101 = 001 = 1
    \item
          5 \(\oplus\) 9 = 0101 \(\oplus\) 1001 = 1100 = 12
\end{itemize}

\hypertarget{redundancy}{%
    \subsection*{1-Redundancy}\label{redundancy}}

The aim of this exercise is to reproduce a RAID-4 software algorithm, by
converting a single input(``document.pdf'') into four data files
``raid4.0'',``raid4.1'',``raid4.2'',``raid4.3'', and one parity file.

\hypertarget{section}{%
    \subsubsection*{1.1}\label{section}}

Starting with the original file and reading bunches of 4 bytes each time
is possible to divide the group between the final files. The parity is
then calculated over each bunch of bytes as D0 \(\oplus\) D1 \(\oplus\)
D2 \(\oplus\) D3,and then printed in the last file.

\begin{tcolorbox}[breakable, size=fbox, boxrule=1pt, pad at break*=1mm,colback=cellbackground, colframe=cellborder]
    \prompt{In}{incolor}{1}{\boxspacing}
    \begin{Verbatim}[commandchars=\\\{\}]
        \PY{n}{filenames} \PY{o}{=} \PY{p}{[}\PY{l+s+s2}{\PYZdq{}}\PY{l+s+s2}{raid4.0}\PY{l+s+s2}{\PYZdq{}}\PY{p}{,}\PY{l+s+s2}{\PYZdq{}}\PY{l+s+s2}{raid4.1}\PY{l+s+s2}{\PYZdq{}}\PY{p}{,}\PY{l+s+s2}{\PYZdq{}}\PY{l+s+s2}{raid4.2}\PY{l+s+s2}{\PYZdq{}}\PY{p}{,}\PY{l+s+s2}{\PYZdq{}}\PY{l+s+s2}{raid4.3}\PY{l+s+s2}{\PYZdq{}}\PY{p}{,}\PY{l+s+s2}{\PYZdq{}}\PY{l+s+s2}{raid4.4}\PY{l+s+s2}{\PYZdq{}}\PY{p}{]}
        \PY{n}{doc} \PY{o}{=} \PY{n+nb}{open}\PY{p}{(}\PY{l+s+s2}{\PYZdq{}}\PY{l+s+s2}{document.pdf}\PY{l+s+s2}{\PYZdq{}}\PY{p}{,} \PY{l+s+s2}{\PYZdq{}}\PY{l+s+s2}{rb}\PY{l+s+s2}{\PYZdq{}}\PY{p}{)}
        \PY{n}{flist} \PY{o}{=} \PY{p}{[}\PY{p}{]}
        \PY{n}{llist} \PY{o}{=} \PY{p}{[}\PY{p}{]}
        \PY{k}{for} \PY{n}{name} \PY{o+ow}{in} \PY{n}{filenames}\PY{p}{:}
        \PY{n}{flist}\PY{o}{.}\PY{n}{append}\PY{p}{(}\PY{n+nb}{open}\PY{p}{(}\PY{n}{name}\PY{p}{,} \PY{l+s+s2}{\PYZdq{}}\PY{l+s+s2}{wb}\PY{l+s+s2}{\PYZdq{}}\PY{p}{)}\PY{p}{)}
        \PY{n}{llist}\PY{o}{.}\PY{n}{append}\PY{p}{(}\PY{p}{[}\PY{p}{]}\PY{p}{)}
        \PY{n}{parities} \PY{o}{=} \PY{p}{[}\PY{l+m+mi}{0}\PY{p}{]}\PY{o}{*}\PY{l+m+mi}{5}
    \end{Verbatim}
\end{tcolorbox}

\begin{tcolorbox}[breakable, size=fbox, boxrule=1pt, pad at break*=1mm,colback=cellbackground, colframe=cellborder]
    \prompt{In}{incolor}{3}{\boxspacing}
    \begin{Verbatim}[commandchars=\\\{\}]
        \PY{n}{byt} \PY{o}{=} \PY{n}{doc}\PY{o}{.}\PY{n}{read}\PY{p}{(}\PY{l+m+mi}{4}\PY{p}{)}
        \PY{k}{while} \PY{n+nb}{len}\PY{p}{(}\PY{n}{byt}\PY{p}{)}\PY{o}{==}\PY{l+m+mi}{4}\PY{p}{:}
        \PY{k}{for} \PY{n}{i} \PY{o+ow}{in} \PY{n+nb}{range}\PY{p}{(}\PY{l+m+mi}{4}\PY{p}{)}\PY{p}{:}
        \PY{n}{flist}\PY{p}{[}\PY{n}{i}\PY{p}{]}\PY{o}{.}\PY{n}{write}\PY{p}{(}\PY{n+nb}{bytes}\PY{p}{(}\PY{p}{[}\PY{n}{byt}\PY{p}{[}\PY{n}{i}\PY{p}{]}\PY{p}{]}\PY{p}{)}\PY{p}{)}
        \PY{n}{llist}\PY{p}{[}\PY{n}{i}\PY{p}{]}\PY{o}{.}\PY{n}{append}\PY{p}{(}\PY{n}{byt}\PY{p}{[}\PY{n}{i}\PY{p}{]}\PY{p}{)}
        \PY{n}{flist}\PY{p}{[}\PY{l+m+mi}{4}\PY{p}{]}\PY{o}{.}\PY{n}{write}\PY{p}{(}\PY{n+nb}{bytes}\PY{p}{(}\PY{p}{[}\PY{n}{byt}\PY{p}{[}\PY{l+m+mi}{0}\PY{p}{]}\PY{o}{\PYZca{}}\PY{n}{byt}\PY{p}{[}\PY{l+m+mi}{1}\PY{p}{]}\PY{o}{\PYZca{}}\PY{n}{byt}\PY{p}{[}\PY{l+m+mi}{2}\PY{p}{]}\PY{o}{\PYZca{}}\PY{n}{byt}\PY{p}{[}\PY{l+m+mi}{3}\PY{p}{]}\PY{p}{]}\PY{p}{)}\PY{p}{)}
        \PY{n}{llist}\PY{p}{[}\PY{l+m+mi}{4}\PY{p}{]}\PY{o}{.}\PY{n}{append}\PY{p}{(}\PY{n}{byt}\PY{p}{[}\PY{l+m+mi}{0}\PY{p}{]}\PY{o}{\PYZca{}}\PY{n}{byt}\PY{p}{[}\PY{l+m+mi}{1}\PY{p}{]}\PY{o}{\PYZca{}}\PY{n}{byt}\PY{p}{[}\PY{l+m+mi}{2}\PY{p}{]}\PY{o}{\PYZca{}}\PY{n}{byt}\PY{p}{[}\PY{l+m+mi}{3}\PY{p}{]}\PY{p}{)}
        \PY{n}{byt} \PY{o}{=} \PY{n}{doc}\PY{o}{.}\PY{n}{read}\PY{p}{(}\PY{l+m+mi}{4}\PY{p}{)}
        \PY{k}{if} \PY{n+nb}{len}\PY{p}{(}\PY{n}{byt}\PY{p}{)}\PY{o}{==}\PY{l+m+mi}{0}\PY{p}{:}
        \PY{k}{break}
        \PY{k}{elif} \PY{n+nb}{len}\PY{p}{(}\PY{n}{byt}\PY{p}{)}\PY{o}{\PYZlt{}}\PY{l+m+mi}{4}\PY{p}{:}                \PY{c+c1}{\PYZsh{}Filling missing bytes using }
        \PY{n}{byt} \PY{o}{=} \PY{n+nb}{list}\PY{p}{(}\PY{n}{byt}\PY{p}{)}             \PY{c+c1}{\PYZsh{}a number of bytes defined as zero}
        \PY{n}{byt}\PY{o}{.}\PY{n}{extend}\PY{p}{(}\PY{p}{[}\PY{l+m+mi}{0}\PY{p}{]}\PY{o}{*}\PY{p}{(}\PY{l+m+mi}{4}\PY{o}{\PYZhy{}}\PY{n+nb}{len}\PY{p}{(}\PY{n}{byt}\PY{p}{)}\PY{p}{)}\PY{p}{)}

    \end{Verbatim}
\end{tcolorbox}

If there are less than 4 bytes, an adequate number of bytes zero is
added in order to apply correctly the \texttt{XOR} operation.

The output files are visible in the folder.

\hypertarget{section}{%
    \subsubsection*{1.2}\label{section}}

\hypertarget{answer}{%
    \paragraph{\texorpdfstring{\textbf{Answer:}}{Answer:}}\label{answer}}

The column-wise parity acts as a checksum for each strip file.

\begin{tcolorbox}[breakable, size=fbox, boxrule=1pt, pad at break*=1mm,colback=cellbackground, colframe=cellborder]
    \prompt{In}{incolor}{4}{\boxspacing}
    \begin{Verbatim}[commandchars=\\\{\}]
        \PY{k}{for} \PY{n}{i}\PY{p}{,}\PY{n}{l} \PY{o+ow}{in} \PY{n+nb}{enumerate}\PY{p}{(}\PY{n}{llist}\PY{p}{)}\PY{p}{:}
        \PY{n}{parities}\PY{p}{[}\PY{n}{i}\PY{p}{]} \PY{o}{=} \PY{p}{(}\PY{n}{reduce}\PY{p}{(}\PY{k}{lambda} \PY{n}{x}\PY{p}{,} \PY{n}{y}\PY{p}{:} \PY{n}{x} \PY{o}{\PYZca{}} \PY{n}{y}\PY{p}{,} \PY{n}{l}\PY{p}{)}\PY{p}{)}

        \PY{k}{for} \PY{n}{fs} \PY{o+ow}{in} \PY{n}{flist}\PY{p}{:}
        \PY{n}{fs}\PY{o}{.}\PY{n}{close}\PY{p}{(}\PY{p}{)}
        \PY{n}{doc}\PY{o}{.}\PY{n}{close}\PY{p}{(}\PY{p}{)}
    \end{Verbatim}
\end{tcolorbox}

\begin{tcolorbox}[breakable, size=fbox, boxrule=1pt, pad at break*=1mm,colback=cellbackground, colframe=cellborder]
    \prompt{In}{incolor}{10}{\boxspacing}
    \begin{Verbatim}[commandchars=\\\{\}]
        \PY{n+nb}{print}\PY{p}{(}\PY{l+s+s2}{\PYZdq{}}\PY{l+s+s2}{The parities values are }\PY{l+s+s2}{\PYZdq{}}\PY{p}{,}\PY{n+nb}{list}\PY{p}{(}\PY{n+nb}{map}\PY{p}{(}\PY{n+nb}{chr}\PY{p}{,}\PY{n}{parities}\PY{p}{)}\PY{p}{)}\PY{p}{)}
    \end{Verbatim}
\end{tcolorbox}

\begin{Verbatim}[commandchars=\\\{\}]
    The parities values are  ['¥', '\textbackslash{}x07', '\textbackslash{}xa0', '\textbackslash{}x9c', '\textbackslash{}x9e']
\end{Verbatim}

The size overhead is calculated as:

\[ \frac{\rm{Total \ size \ of \ the \ final \ files} - \rm{Size \ of \ the \ original \ file}}{ \rm{Size \ of \ the \ original  \ file}} \]

With the size of the original file of 170619 bytes.

\hypertarget{answer}{%
    \paragraph{\texorpdfstring{\textbf{Answer:}}{Answer:}}\label{answer}}

\begin{tcolorbox}[breakable, size=fbox, boxrule=1pt, pad at break*=1mm,colback=cellbackground, colframe=cellborder]
    \prompt{In}{incolor}{8}{\boxspacing}
    \begin{Verbatim}[commandchars=\\\{\}]
        \PY{n+nb}{print}\PY{p}{(}\PY{l+s+sa}{f}\PY{l+s+s2}{\PYZdq{}}\PY{l+s+s2}{The size overhead is }\PY{l+s+si}{\PYZob{}}\PY{p}{(}\PY{n+nb}{sum}\PY{p}{(}\PY{n+nb}{list}\PY{p}{(}\PY{n+nb}{map}\PY{p}{(}\PY{n+nb}{len}\PY{p}{,}\PY{n}{llist}\PY{p}{)}\PY{p}{)}\PY{p}{)}\PY{o}{\PYZhy{}}\PY{l+m+mi}{170619}\PY{p}{)}\PY{o}{/}\PY{p}{(}\PY{l+m+mi}{170619}\PY{p}{)}\PY{o}{*}\PY{l+m+mi}{100}\PY{l+s+si}{:}\PY{l+s+s2}{.2f}\PY{l+s+si}{\PYZcb{}}\PY{l+s+s2}{\PYZpc{}}\PY{l+s+s2}{\PYZdq{}}\PY{p}{)}
    \end{Verbatim}
\end{tcolorbox}

\begin{Verbatim}[commandchars=\\\{\}]
    The size overhead is 25.00\%
\end{Verbatim}

\hypertarget{section}{%
    \subsubsection*{1.3}\label{section}}

\begin{tcolorbox}[breakable, size=fbox, boxrule=1pt, pad at break*=1mm,colback=cellbackground, colframe=cellborder]
    \prompt{In}{incolor}{11}{\boxspacing}
    \begin{Verbatim}[commandchars=\\\{\}]
        \PY{n+nb}{print}\PY{p}{(}\PY{l+s+sa}{f}\PY{l+s+s2}{\PYZdq{}}\PY{l+s+s2}{The 5 bytes parity value is }\PY{l+s+si}{\PYZob{}}\PY{n+nb}{hex}\PY{p}{(}\PY{n}{parities}\PY{p}{[}\PY{l+m+mi}{0}\PY{p}{]}\PY{p}{)}\PY{l+s+si}{\PYZcb{}}\PY{l+s+s2}{0}\PY{l+s+si}{\PYZob{}}\PY{n+nb}{hex}\PY{p}{(}\PY{n}{parities}\PY{p}{[}\PY{l+m+mi}{1}\PY{p}{]}\PY{p}{)}\PY{p}{[}\PY{l+m+mi}{2}\PY{p}{:}\PY{l+m+mi}{4}\PY{p}{]}\PY{l+s+si}{\PYZcb{}}\PY{l+s+si}{\PYZob{}}\PY{n+nb}{hex}\PY{p}{(}\PY{n}{parities}\PY{p}{[}\PY{l+m+mi}{2}\PY{p}{]}\PY{p}{)}\PY{p}{[}\PY{l+m+mi}{2}\PY{p}{:}\PY{l+m+mi}{4}\PY{p}{]}\PY{l+s+si}{\PYZcb{}}\PY{l+s+si}{\PYZob{}}\PY{n+nb}{hex}\PY{p}{(}\PY{n}{parities}\PY{p}{[}\PY{l+m+mi}{3}\PY{p}{]}\PY{p}{)}\PY{p}{[}\PY{l+m+mi}{2}\PY{p}{:}\PY{l+m+mi}{4}\PY{p}{]}\PY{l+s+si}{\PYZcb{}}\PY{l+s+si}{\PYZob{}}\PY{n+nb}{hex}\PY{p}{(}\PY{n}{parities}\PY{p}{[}\PY{l+m+mi}{4}\PY{p}{]}\PY{p}{)}\PY{p}{[}\PY{l+m+mi}{2}\PY{p}{:}\PY{l+m+mi}{4}\PY{p}{]}\PY{l+s+si}{\PYZcb{}}\PY{l+s+s2}{\PYZdq{}}\PY{p}{)}
    \end{Verbatim}
\end{tcolorbox}

\begin{Verbatim}[commandchars=\\\{\}]
    The 5 bytes parity value is 0xa507a09c9e
\end{Verbatim}

\hypertarget{section}{%
    \subsubsection*{1.4}\label{section}}

The row-wise parities of the five stripe files would be zero since the
operation consists in a \texttt{XOR} between two identical quantities.
In this way the resulting file will be filled with bytes zero.

D0 \(\oplus\) D1\(\oplus\)D2\(\oplus\)D3 \(\oplus\) P=(D0 \(\oplus\)
D1\(\oplus\)D2\(\oplus\)D3)\(\oplus\)(D0\(\oplus\)D1\(\oplus\)D2\(\oplus\)D3)
= (D0\(\oplus\)D0) \(\oplus\) (D1\(\oplus\)D1) \(\oplus\)
(D2\(\oplus\)D2) \(\oplus\) (D3\(\oplus\)D3) = 0

Result that is due to the properties of the \texttt{XOR} operation given
P as the byte of the parity file.

\hypertarget{section}{%
    \subsubsection*{1.5}\label{section}}

Since the new parity is 0xa507a0{\color{red}01}9e, then the error is located in the
fourth disk. In order to reconstruct the original content, the files to
be used are the first three and the parity one. By applying the parity
between each group of bytes coming from these files the result will be
the content of the failed disk.

The restruction operation per byte is D3 = D0 \(\oplus\)
D1\(\oplus\)D2\(\oplus\)P

\hypertarget{section}{%
    \subsection*{2}\label{section}}

\hypertarget{section}{%
    \subsubsection*{2.1}\label{section}}

The encryption technique is clearly a symmetric one since the key is the
same used to encrypt and decrypt the message.

\hypertarget{section}{%
    \subsubsection*{2.2}\label{section}}

The original message is ``K{]}amua!trgpy''. The decryption will follow
the inverse process with respect to the encryption one, starting from
the nonce removal and following with the key removal. In this way the
operations are reverted and is possbile to find the correct sentence.

\begin{tcolorbox}[breakable, size=fbox, boxrule=1pt, pad at break*=1mm,colback=cellbackground, colframe=cellborder]
    \prompt{In}{incolor}{63}{\boxspacing}
    \begin{Verbatim}[commandchars=\\\{\}]
        \PY{n}{alist} \PY{o}{=} \PY{n}{np}\PY{o}{.}\PY{n}{asarray}\PY{p}{(}\PY{n+nb}{list}\PY{p}{(}\PY{n+nb}{map}\PY{p}{(}\PY{n+nb}{ord}\PY{p}{,}\PY{l+s+s2}{\PYZdq{}}\PY{l+s+s2}{K]amua!trgpy}\PY{l+s+s2}{\PYZdq{}}\PY{p}{)}\PY{p}{)}\PY{p}{)}
    \end{Verbatim}
\end{tcolorbox}

The original string is transformed into an array of ASCII values.

\begin{tcolorbox}[breakable, size=fbox, boxrule=1pt, pad at break*=1mm,colback=cellbackground, colframe=cellborder]
    \prompt{In}{incolor}{64}{\boxspacing}
    \begin{Verbatim}[commandchars=\\\{\}]
        \PY{n}{alist} \PY{o}{=} \PY{n}{alist} \PY{o}{\PYZhy{}} \PY{n}{np}\PY{o}{.}\PY{n}{arange}\PY{p}{(}\PY{l+m+mi}{5}\PY{p}{,}\PY{l+m+mi}{5}\PY{o}{+}\PY{n}{alist}\PY{o}{.}\PY{n}{shape}\PY{p}{[}\PY{l+m+mi}{0}\PY{p}{]}\PY{p}{)} \PY{c+c1}{\PYZsh{}\PYZsh{} Removing the nonce}
        \PY{n}{clist} \PY{o}{=} \PY{n}{alist}\PY{o}{.}\PY{n}{copy}\PY{p}{(}\PY{p}{)}\PY{o}{.}\PY{n}{reshape}\PY{p}{(}\PY{l+m+mi}{1}\PY{p}{,}\PY{o}{\PYZhy{}}\PY{l+m+mi}{1}\PY{p}{)}
    \end{Verbatim}
\end{tcolorbox}

The nonce is removed using numpy functions.

\begin{tcolorbox}[breakable, size=fbox, boxrule=1pt, pad at break*=1mm,colback=cellbackground, colframe=cellborder]
    \prompt{In}{incolor}{65}{\boxspacing}
    \begin{Verbatim}[commandchars=\\\{\}]
        \PY{k}{for} \PY{n}{i} \PY{o+ow}{in} \PY{n+nb}{range}\PY{p}{(}\PY{l+m+mi}{1}\PY{p}{,}\PY{l+m+mi}{256}\PY{p}{)}\PY{p}{:}
        \PY{k}{if} \PY{n}{np}\PY{o}{.}\PY{n}{all}\PY{p}{(}\PY{p}{(}\PY{n}{alist} \PY{o}{\PYZhy{}} \PY{n}{i}\PY{p}{)}\PY{o}{\PYZgt{}}\PY{l+m+mi}{0}\PY{p}{)}\PY{p}{:}
        \PY{n}{clist} \PY{o}{=} \PY{n}{np}\PY{o}{.}\PY{n}{concatenate}\PY{p}{(}\PY{p}{[}\PY{n}{clist}\PY{p}{,}\PY{p}{(}\PY{n}{alist} \PY{o}{\PYZhy{}} \PY{n}{i}\PY{p}{)}\PY{o}{.}\PY{n}{reshape}\PY{p}{(}\PY{l+m+mi}{1}\PY{p}{,}\PY{o}{\PYZhy{}}\PY{l+m+mi}{1}\PY{p}{)}\PY{p}{]}\PY{p}{)} \PY{c+c1}{\PYZsh{}\PYZsh{} Decryption attempts}
        \PY{k}{else}\PY{p}{:}
        \PY{n}{nlist} \PY{o}{=} \PY{p}{(}\PY{n}{alist} \PY{o}{\PYZhy{}} \PY{n}{i}\PY{p}{)}
        \PY{n}{nlist}\PY{p}{[}\PY{n}{nlist} \PY{o}{\PYZlt{}}\PY{l+m+mi}{0}\PY{p}{]}\PY{o}{+}\PY{o}{=}\PY{l+m+mi}{255}
        \PY{n}{clist} \PY{o}{=} \PY{n}{np}\PY{o}{.}\PY{n}{concatenate}\PY{p}{(}\PY{p}{[}\PY{n}{clist}\PY{p}{,}\PY{n}{nlist}\PY{o}{.}\PY{n}{reshape}\PY{p}{(}\PY{l+m+mi}{1}\PY{p}{,}\PY{o}{\PYZhy{}}\PY{l+m+mi}{1}\PY{p}{)}\PY{p}{]}\PY{p}{)}
    \end{Verbatim}
\end{tcolorbox}

In the same way all the numbers between 1 and 255 are subtracted to the
original message in order to find the correct results.

Since with large numbers the resulting array contains negative numbers,
in this case they are increased by 255, emulating a periodic set of
numbers. This is the only way to get a significative result.

\begin{tcolorbox}[breakable, size=fbox, boxrule=1pt, pad at break*=1mm,colback=cellbackground, colframe=cellborder]
    \prompt{In}{incolor}{67}{\boxspacing}
    \begin{Verbatim}[commandchars=\\\{\}]
        \PY{n}{wordlist} \PY{o}{=} \PY{p}{[}\PY{p}{[}\PY{l+s+s1}{\PYZsq{}}\PY{l+s+s1}{\PYZsq{}}\PY{o}{.}\PY{n}{join}\PY{p}{(}\PY{p}{[}\PY{n+nb}{chr}\PY{p}{(}\PY{n}{x}\PY{p}{)} \PY{k}{for} \PY{n}{x} \PY{o+ow}{in} \PY{n}{word}\PY{p}{]}\PY{p}{)}\PY{p}{,}\PY{n}{i}\PY{o}{+}\PY{l+m+mi}{1}\PY{p}{]} \PY{k}{for} \PY{n}{i}\PY{p}{,}\PY{n}{word} \PY{o+ow}{in} \PY{n+nb}{enumerate}\PY{p}{(}\PY{n}{clist}\PY{p}{)}\PY{p}{]}
    \end{Verbatim}
\end{tcolorbox}

\hypertarget{answer}{%
    \paragraph{\texorpdfstring{\textbf{Answer:}}{Answer:}}\label{answer}}

\begin{tcolorbox}[breakable, size=fbox, boxrule=1pt, pad at break*=1mm,colback=cellbackground, colframe=cellborder]
    \prompt{In}{incolor}{70}{\boxspacing}
    \begin{Verbatim}[commandchars=\\\{\}]
        \PY{n+nb}{print}\PY{p}{(}\PY{l+s+sa}{f}\PY{l+s+s2}{\PYZdq{}}\PY{l+s+s2}{The used key is }\PY{l+s+si}{\PYZob{}}\PY{n}{wordlist}\PY{p}{[}\PY{l+m+mi}{245}\PY{p}{]}\PY{p}{[}\PY{l+m+mi}{1}\PY{p}{]}\PY{l+s+si}{\PYZcb{}}\PY{l+s+s2}{, the original message text is }\PY{l+s+se}{\PYZbs{}\PYZdq{}}\PY{l+s+si}{\PYZob{}}\PY{n}{wordlist}\PY{p}{[}\PY{l+m+mi}{245}\PY{p}{]}\PY{p}{[}\PY{l+m+mi}{0}\PY{p}{]}\PY{l+s+si}{\PYZcb{}}\PY{l+s+se}{\PYZbs{}\PYZdq{}}\PY{l+s+s2}{\PYZdq{}}\PY{p}{)}
    \end{Verbatim}
\end{tcolorbox}

\begin{Verbatim}[commandchars=\\\{\}]
    The used key is 246, the original message text is "Padova rocks"
\end{Verbatim}

\hypertarget{section}{%
    \subsection*{3}\label{section}}

\hypertarget{section}{%
    \subsubsection*{3.1}\label{section}}

The occupation process can be simulated using random functions given by
numpy.

\begin{tcolorbox}[breakable, size=fbox, boxrule=1pt, pad at break*=1mm,colback=cellbackground, colframe=cellborder]
    \prompt{In}{incolor}{11}{\boxspacing}
    \begin{Verbatim}[commandchars=\\\{\}]
        \PY{n}{occ10} \PY{o}{=} \PY{n}{np}\PY{o}{.}\PY{n}{zeros}\PY{p}{(}\PY{l+m+mi}{10}\PY{p}{)}

        \PY{k}{while}  \PY{p}{(}\PY{o+ow}{not} \PY{n}{np}\PY{o}{.}\PY{n}{any}\PY{p}{(}\PY{n}{occ10} \PY{o}{\PYZgt{}}\PY{o}{=} \PY{l+m+mf}{1.}\PY{p}{)}\PY{p}{)}\PY{p}{:}
        \PY{n}{occ10}\PY{p}{[}\PY{n}{np}\PY{o}{.}\PY{n}{random}\PY{o}{.}\PY{n}{randint}\PY{p}{(}\PY{l+m+mi}{0}\PY{p}{,}\PY{l+m+mi}{10}\PY{p}{)}\PY{p}{]}\PY{o}{+}\PY{o}{=}\PY{l+m+mf}{0.01}
    \end{Verbatim}
\end{tcolorbox}

Since each file occupies 10 GB, or 0.01 TB, for each hard disk, if
chosen, the stored memory increases by 0.01.

\begin{tcolorbox}[breakable, size=fbox, boxrule=1pt, pad at break*=1mm,colback=cellbackground, colframe=cellborder]
    \prompt{In}{incolor}{28}{\boxspacing}
    \begin{Verbatim}[commandchars=\\\{\}]
        \PY{n}{fig}\PY{p}{,}\PY{n}{ax} \PY{o}{=} \PY{n}{plt}\PY{o}{.}\PY{n}{subplots}\PY{p}{(}\PY{l+m+mi}{1}\PY{p}{,}\PY{l+m+mi}{1}\PY{p}{,} \PY{n}{figsize} \PY{o}{=} \PY{p}{(}\PY{l+m+mi}{7}\PY{p}{,}\PY{l+m+mi}{5}\PY{p}{)}\PY{p}{)}
        \PY{n}{ax}\PY{o}{.}\PY{n}{bar}\PY{p}{(}\PY{n}{np}\PY{o}{.}\PY{n}{arange}\PY{p}{(}\PY{l+m+mi}{1}\PY{p}{,}\PY{l+m+mi}{11}\PY{p}{)}\PY{p}{,}\PY{n}{occ10}\PY{p}{)}
        \PY{n}{ax}\PY{o}{.}\PY{n}{set\PYZus{}xlabel}\PY{p}{(}\PY{l+s+s2}{\PYZdq{}}\PY{l+s+s2}{Disk number}\PY{l+s+s2}{\PYZdq{}}\PY{p}{)}
        \PY{n}{ax}\PY{o}{.}\PY{n}{set\PYZus{}ylabel}\PY{p}{(}\PY{l+s+s2}{\PYZdq{}}\PY{l+s+s2}{Used memory[TB]}\PY{l+s+s2}{\PYZdq{}}\PY{p}{)}
        \PY{n}{ax}\PY{o}{.}\PY{n}{set\PYZus{}xticks}\PY{p}{(}\PY{n+nb}{list}\PY{p}{(}\PY{n+nb}{range}\PY{p}{(}\PY{l+m+mi}{1}\PY{p}{,}\PY{l+m+mi}{11}\PY{p}{)}\PY{p}{)}\PY{p}{)}
        \PY{n}{plt}\PY{o}{.}\PY{n}{show}\PY{p}{(}\PY{p}{)}
    \end{Verbatim}
\end{tcolorbox}

\begin{center}
    \adjustimage{max size={0.9\linewidth}{0.9\paperheight}}{Final_assignment_files/Final_assignment_45_0.png}
\end{center}
{ \hspace*{\fill} \\}

\hypertarget{a}{%
    \paragraph{3.1a}\label{a}}

\hypertarget{answer}{%
    \subparagraph{\texorpdfstring{\textbf{Answer:}}{Answer:}}\label{answer}}

\begin{tcolorbox}[breakable, size=fbox, boxrule=1pt, pad at break*=1mm,colback=cellbackground, colframe=cellborder]
    \prompt{In}{incolor}{13}{\boxspacing}
    \begin{Verbatim}[commandchars=\\\{\}]
        \PY{n+nb}{print}\PY{p}{(}\PY{l+s+sa}{f}\PY{l+s+s2}{\PYZdq{}}\PY{l+s+s2}{The number of placed files is }\PY{l+s+si}{\PYZob{}}\PY{n+nb}{sum}\PY{p}{(}\PY{n}{occ10}\PY{p}{)}\PY{o}{/}\PY{l+m+mf}{0.01}\PY{l+s+si}{:}\PY{l+s+s2}{.0f}\PY{l+s+si}{\PYZcb{}}\PY{l+s+s2}{\PYZdq{}}\PY{p}{)}
    \end{Verbatim}
\end{tcolorbox}

\begin{Verbatim}[commandchars=\\\{\}]
    The number of placed files is 822
\end{Verbatim}

\hypertarget{b}{%
    \paragraph{3.1b}\label{b}}

\hypertarget{answer}{%
    \subparagraph{\texorpdfstring{\textbf{Answer:}}{Answer:}}\label{answer}}

\begin{tcolorbox}[breakable, size=fbox, boxrule=1pt, pad at break*=1mm,colback=cellbackground, colframe=cellborder]
    \prompt{In}{incolor}{14}{\boxspacing}
    \begin{Verbatim}[commandchars=\\\{\}]
        \PY{n+nb}{print}\PY{p}{(}\PY{l+s+sa}{f}\PY{l+s+s2}{\PYZdq{}}\PY{l+s+s2}{The used space percentage is  }\PY{l+s+si}{\PYZob{}}\PY{n+nb}{sum}\PY{p}{(}\PY{n}{occ10}\PY{p}{)}\PY{o}{*}\PY{l+m+mi}{100}\PY{o}{/}\PY{l+m+mi}{10}\PY{l+s+si}{:}\PY{l+s+s2}{.1f}\PY{l+s+si}{\PYZcb{}}\PY{l+s+s2}{ \PYZpc{}}\PY{l+s+s2}{\PYZdq{}}\PY{p}{)}
    \end{Verbatim}
\end{tcolorbox}

\begin{Verbatim}[commandchars=\\\{\}]
    The used space percentage is  82.2 \%
\end{Verbatim}

\hypertarget{section}{%
    \subsubsection*{3.2}\label{section}}

\begin{tcolorbox}[breakable, size=fbox, boxrule=1pt, pad at break*=1mm,colback=cellbackground, colframe=cellborder]
    \prompt{In}{incolor}{16}{\boxspacing}
    \begin{Verbatim}[commandchars=\\\{\}]
        \PY{n}{occ1} \PY{o}{=} \PY{n}{np}\PY{o}{.}\PY{n}{zeros}\PY{p}{(}\PY{l+m+mi}{10}\PY{p}{)}

        \PY{k}{while}  \PY{p}{(}\PY{o+ow}{not} \PY{n}{np}\PY{o}{.}\PY{n}{any}\PY{p}{(}\PY{n}{occ1} \PY{o}{\PYZgt{}}\PY{o}{=} \PY{l+m+mf}{1.}\PY{p}{)}\PY{p}{)}\PY{p}{:}
        \PY{n}{occ1}\PY{p}{[}\PY{n}{np}\PY{o}{.}\PY{n}{random}\PY{o}{.}\PY{n}{randint}\PY{p}{(}\PY{l+m+mi}{0}\PY{p}{,}\PY{l+m+mi}{10}\PY{p}{)}\PY{p}{]}\PY{o}{+}\PY{o}{=}\PY{l+m+mf}{0.001}
    \end{Verbatim}
\end{tcolorbox}

Since each file occupies 1 GB, or 0.001 TB, for each hard disk, if
chosen, the stored memory increases by 0.001.

\begin{tcolorbox}[breakable, size=fbox, boxrule=1pt, pad at break*=1mm,colback=cellbackground, colframe=cellborder]
    \prompt{In}{incolor}{27}{\boxspacing}
    \begin{Verbatim}[commandchars=\\\{\}]
        \PY{n}{fig}\PY{p}{,}\PY{n}{ax} \PY{o}{=} \PY{n}{plt}\PY{o}{.}\PY{n}{subplots}\PY{p}{(}\PY{l+m+mi}{1}\PY{p}{,}\PY{l+m+mi}{1}\PY{p}{,} \PY{n}{figsize} \PY{o}{=} \PY{p}{(}\PY{l+m+mi}{7}\PY{p}{,}\PY{l+m+mi}{5}\PY{p}{)}\PY{p}{)}
        \PY{n}{ax}\PY{o}{.}\PY{n}{bar}\PY{p}{(}\PY{n}{np}\PY{o}{.}\PY{n}{arange}\PY{p}{(}\PY{l+m+mi}{1}\PY{p}{,}\PY{l+m+mi}{11}\PY{p}{)}\PY{p}{,}\PY{n}{occ1}\PY{p}{)}
        \PY{n}{ax}\PY{o}{.}\PY{n}{set\PYZus{}xlabel}\PY{p}{(}\PY{l+s+s2}{\PYZdq{}}\PY{l+s+s2}{Disk number}\PY{l+s+s2}{\PYZdq{}}\PY{p}{)}
        \PY{n}{ax}\PY{o}{.}\PY{n}{set\PYZus{}ylabel}\PY{p}{(}\PY{l+s+s2}{\PYZdq{}}\PY{l+s+s2}{Used memory[TB]}\PY{l+s+s2}{\PYZdq{}}\PY{p}{)}
        \PY{n}{ax}\PY{o}{.}\PY{n}{set\PYZus{}xticks}\PY{p}{(}\PY{n+nb}{list}\PY{p}{(}\PY{n+nb}{range}\PY{p}{(}\PY{l+m+mi}{1}\PY{p}{,}\PY{l+m+mi}{11}\PY{p}{)}\PY{p}{)}\PY{p}{)}
        \PY{n}{plt}\PY{o}{.}\PY{n}{show}\PY{p}{(}\PY{p}{)}
    \end{Verbatim}
\end{tcolorbox}

\begin{center}
    \adjustimage{max size={0.9\linewidth}{0.9\paperheight}}{Final_assignment_files/Final_assignment_53_0.png}
\end{center}
{ \hspace*{\fill} \\}

\hypertarget{a}{%
    \paragraph{3.2a}\label{a}}

\hypertarget{answer}{%
    \subparagraph{\texorpdfstring{\textbf{Answer:}}{Answer:}}\label{answer}}

\begin{tcolorbox}[breakable, size=fbox, boxrule=1pt, pad at break*=1mm,colback=cellbackground, colframe=cellborder]
    \prompt{In}{incolor}{20}{\boxspacing}
    \begin{Verbatim}[commandchars=\\\{\}]
        \PY{n+nb}{print}\PY{p}{(}\PY{l+s+sa}{f}\PY{l+s+s2}{\PYZdq{}}\PY{l+s+s2}{The number of placed files is }\PY{l+s+si}{\PYZob{}}\PY{n+nb}{sum}\PY{p}{(}\PY{n}{occ1}\PY{p}{)}\PY{o}{/}\PY{l+m+mf}{0.001}\PY{l+s+si}{:}\PY{l+s+s2}{.0f}\PY{l+s+si}{\PYZcb{}}\PY{l+s+s2}{\PYZdq{}}\PY{p}{)}
    \end{Verbatim}
\end{tcolorbox}

\begin{Verbatim}[commandchars=\\\{\}]
    The number of placed files is 9755
\end{Verbatim}

\hypertarget{b}{%
    \paragraph{3.2b}\label{b}}

\hypertarget{answer}{%
    \subparagraph{\texorpdfstring{\textbf{Answer:}}{Answer:}}\label{answer}}

\begin{tcolorbox}[breakable, size=fbox, boxrule=1pt, pad at break*=1mm,colback=cellbackground, colframe=cellborder]
    \prompt{In}{incolor}{21}{\boxspacing}
    \begin{Verbatim}[commandchars=\\\{\}]
        \PY{n+nb}{print}\PY{p}{(}\PY{l+s+sa}{f}\PY{l+s+s2}{\PYZdq{}}\PY{l+s+s2}{The used space percentage is  }\PY{l+s+si}{\PYZob{}}\PY{n+nb}{sum}\PY{p}{(}\PY{n}{occ1}\PY{p}{)}\PY{o}{*}\PY{l+m+mi}{100}\PY{o}{/}\PY{l+m+mi}{10}\PY{l+s+si}{:}\PY{l+s+s2}{.2f}\PY{l+s+si}{\PYZcb{}}\PY{l+s+s2}{ \PYZpc{}}\PY{l+s+s2}{\PYZdq{}}\PY{p}{)}
    \end{Verbatim}
\end{tcolorbox}

\begin{Verbatim}[commandchars=\\\{\}]
    The used space percentage is  97.55 \%
\end{Verbatim}

\hypertarget{section}{%
    \subsubsection*{3.3}\label{section}}

From these two examples is clear that by reducing the chunk size a
better space occupation is possible. Files are usually stored in 4M
chunks, since it allows a better redistribution of files over all the
memories.

\begin{tcolorbox}[breakable, size=fbox, boxrule=1pt, pad at break*=1mm,colback=cellbackground, colframe=cellborder]
    \prompt{In}{incolor}{22}{\boxspacing}
    \begin{Verbatim}[commandchars=\\\{\}]
        \PY{n}{occ4} \PY{o}{=} \PY{n}{np}\PY{o}{.}\PY{n}{zeros}\PY{p}{(}\PY{l+m+mi}{10}\PY{p}{)}
        \PY{k}{while}  \PY{p}{(}\PY{o+ow}{not} \PY{n}{np}\PY{o}{.}\PY{n}{any}\PY{p}{(}\PY{n}{occ4} \PY{o}{\PYZgt{}}\PY{o}{=} \PY{l+m+mf}{1.}\PY{p}{)}\PY{p}{)}\PY{p}{:}
        \PY{n}{occ4}\PY{p}{[}\PY{n}{np}\PY{o}{.}\PY{n}{random}\PY{o}{.}\PY{n}{randint}\PY{p}{(}\PY{l+m+mi}{0}\PY{p}{,}\PY{l+m+mi}{10}\PY{p}{)}\PY{p}{]}\PY{o}{+}\PY{o}{=}\PY{l+m+mf}{4e\PYZhy{}6}
    \end{Verbatim}
\end{tcolorbox}

Since each file occupies 4 MB, or \(4\cdot 10^{-6}\) TB, for each hard
disk, if chosen, the stored memory increases by \(4\cdot 10^{-6}\).

\begin{tcolorbox}[breakable, size=fbox, boxrule=1pt, pad at break*=1mm,colback=cellbackground, colframe=cellborder]
    \prompt{In}{incolor}{26}{\boxspacing}
    \begin{Verbatim}[commandchars=\\\{\}]
        \PY{n}{fig}\PY{p}{,}\PY{n}{ax} \PY{o}{=} \PY{n}{plt}\PY{o}{.}\PY{n}{subplots}\PY{p}{(}\PY{l+m+mi}{1}\PY{p}{,}\PY{l+m+mi}{1}\PY{p}{,} \PY{n}{figsize} \PY{o}{=} \PY{p}{(}\PY{l+m+mi}{7}\PY{p}{,}\PY{l+m+mi}{5}\PY{p}{)}\PY{p}{)}
        \PY{n}{ax}\PY{o}{.}\PY{n}{bar}\PY{p}{(}\PY{n}{np}\PY{o}{.}\PY{n}{arange}\PY{p}{(}\PY{l+m+mi}{1}\PY{p}{,}\PY{l+m+mi}{11}\PY{p}{)}\PY{p}{,}\PY{n}{occ4}\PY{p}{)}
        \PY{n}{ax}\PY{o}{.}\PY{n}{set\PYZus{}xlabel}\PY{p}{(}\PY{l+s+s2}{\PYZdq{}}\PY{l+s+s2}{Disk number}\PY{l+s+s2}{\PYZdq{}}\PY{p}{)}
        \PY{n}{ax}\PY{o}{.}\PY{n}{set\PYZus{}ylabel}\PY{p}{(}\PY{l+s+s2}{\PYZdq{}}\PY{l+s+s2}{Used memory[TB]}\PY{l+s+s2}{\PYZdq{}}\PY{p}{)}
        \PY{n}{ax}\PY{o}{.}\PY{n}{set\PYZus{}xticks}\PY{p}{(}\PY{n+nb}{list}\PY{p}{(}\PY{n+nb}{range}\PY{p}{(}\PY{l+m+mi}{1}\PY{p}{,}\PY{l+m+mi}{11}\PY{p}{)}\PY{p}{)}\PY{p}{)}
        \PY{n}{plt}\PY{o}{.}\PY{n}{show}\PY{p}{(}\PY{p}{)}
    \end{Verbatim}
\end{tcolorbox}

\begin{center}
    \adjustimage{max size={0.9\linewidth}{0.9\paperheight}}{Final_assignment_files/Final_assignment_62_0.png}
\end{center}
{ \hspace*{\fill} \\}

\begin{tcolorbox}[breakable, size=fbox, boxrule=1pt, pad at break*=1mm,colback=cellbackground, colframe=cellborder]
    \prompt{In}{incolor}{24}{\boxspacing}
    \begin{Verbatim}[commandchars=\\\{\}]
        \PY{n+nb}{print}\PY{p}{(}\PY{l+s+sa}{f}\PY{l+s+s2}{\PYZdq{}}\PY{l+s+s2}{The number of placed files is }\PY{l+s+si}{\PYZob{}}\PY{n+nb}{sum}\PY{p}{(}\PY{n}{occ4}\PY{p}{)}\PY{o}{/}\PY{l+m+mf}{4e\PYZhy{}6}\PY{l+s+si}{:}\PY{l+s+s2}{.0f}\PY{l+s+si}{\PYZcb{}}\PY{l+s+s2}{\PYZdq{}}\PY{p}{)}
    \end{Verbatim}
\end{tcolorbox}

\begin{Verbatim}[commandchars=\\\{\}]
    The number of placed files is 2490571
\end{Verbatim}

\begin{tcolorbox}[breakable, size=fbox, boxrule=1pt, pad at break*=1mm,colback=cellbackground, colframe=cellborder]
    \prompt{In}{incolor}{25}{\boxspacing}
    \begin{Verbatim}[commandchars=\\\{\}]
        \PY{n+nb}{print}\PY{p}{(}\PY{l+s+sa}{f}\PY{l+s+s2}{\PYZdq{}}\PY{l+s+s2}{The occupation percentage is  }\PY{l+s+si}{\PYZob{}}\PY{n+nb}{sum}\PY{p}{(}\PY{n}{occ4}\PY{p}{)}\PY{o}{*}\PY{l+m+mi}{100}\PY{o}{/}\PY{l+m+mi}{10}\PY{l+s+si}{:}\PY{l+s+s2}{.2f}\PY{l+s+si}{\PYZcb{}}\PY{l+s+s2}{ \PYZpc{}}\PY{l+s+s2}{\PYZdq{}}\PY{p}{)}
    \end{Verbatim}
\end{tcolorbox}

\begin{Verbatim}[commandchars=\\\{\}]
    The occupation percentage is  99.62 \%
\end{Verbatim}

The benefit is clear, since almost all the available space is used.

\hypertarget{section}{%
    \subsubsection*{3.4}\label{section}}

\begin{tcolorbox}[breakable, size=fbox, boxrule=1pt, pad at break*=1mm,colback=cellbackground, colframe=cellborder]
    \prompt{In}{incolor}{38}{\boxspacing}
    \begin{Verbatim}[commandchars=\\\{\}]
        \PY{n+nb}{print}\PY{p}{(}\PY{l+s+sa}{f}\PY{l+s+s2}{\PYZdq{}}\PY{l+s+s2}{The average used space for the first case is }\PY{l+s+si}{\PYZob{}}\PY{n}{np}\PY{o}{.}\PY{n}{mean}\PY{p}{(}\PY{n}{occ10}\PY{p}{)}\PY{l+s+si}{:}\PY{l+s+s2}{.3f}\PY{l+s+si}{\PYZcb{}}\PY{l+s+s2}{ TB with a standard deviation of }\PY{l+s+si}{\PYZob{}}\PY{n}{np}\PY{o}{.}\PY{n}{std}\PY{p}{(}\PY{n}{occ10}\PY{p}{)}\PY{l+s+si}{:}\PY{l+s+s2}{.3f}\PY{l+s+si}{\PYZcb{}}\PY{l+s+s2}{ TB}\PY{l+s+s2}{\PYZdq{}}\PY{p}{)}
    \end{Verbatim}
\end{tcolorbox}

\begin{Verbatim}[commandchars=\\\{\}]
    The average used space for the first case is 0.864 TB with a standard deviation
    of 0.088 TB
\end{Verbatim}

\begin{tcolorbox}[breakable, size=fbox, boxrule=1pt, pad at break*=1mm,colback=cellbackground, colframe=cellborder]
    \prompt{In}{incolor}{39}{\boxspacing}
    \begin{Verbatim}[commandchars=\\\{\}]
        \PY{n+nb}{print}\PY{p}{(}\PY{l+s+sa}{f}\PY{l+s+s2}{\PYZdq{}}\PY{l+s+s2}{The average used space for the second case is }\PY{l+s+si}{\PYZob{}}\PY{n}{np}\PY{o}{.}\PY{n}{mean}\PY{p}{(}\PY{n}{occ1}\PY{p}{)}\PY{l+s+si}{:}\PY{l+s+s2}{.3f}\PY{l+s+si}{\PYZcb{}}\PY{l+s+s2}{ TB with a standard deviation of }\PY{l+s+si}{\PYZob{}}\PY{n}{np}\PY{o}{.}\PY{n}{std}\PY{p}{(}\PY{n}{occ1}\PY{p}{)}\PY{l+s+si}{:}\PY{l+s+s2}{.3f}\PY{l+s+si}{\PYZcb{}}\PY{l+s+s2}{ TB}\PY{l+s+s2}{\PYZdq{}}\PY{p}{)}
    \end{Verbatim}
\end{tcolorbox}

\begin{Verbatim}[commandchars=\\\{\}]
    The average used space for the second case is 0.954 TB with a standard deviation
    of 0.031 TB
\end{Verbatim}

\begin{tcolorbox}[breakable, size=fbox, boxrule=1pt, pad at break*=1mm,colback=cellbackground, colframe=cellborder]
    \prompt{In}{incolor}{40}{\boxspacing}
    \begin{Verbatim}[commandchars=\\\{\}]
        \PY{n+nb}{print}\PY{p}{(}\PY{l+s+sa}{f}\PY{l+s+s2}{\PYZdq{}}\PY{l+s+s2}{The average used space for the third case is }\PY{l+s+si}{\PYZob{}}\PY{n}{np}\PY{o}{.}\PY{n}{mean}\PY{p}{(}\PY{n}{occ4}\PY{p}{)}\PY{l+s+si}{:}\PY{l+s+s2}{.3f}\PY{l+s+si}{\PYZcb{}}\PY{l+s+s2}{ TB with a standard deviation of }\PY{l+s+si}{\PYZob{}}\PY{n}{np}\PY{o}{.}\PY{n}{std}\PY{p}{(}\PY{n}{occ4}\PY{p}{)}\PY{l+s+si}{:}\PY{l+s+s2}{.3f}\PY{l+s+si}{\PYZcb{}}\PY{l+s+s2}{ TB}\PY{l+s+s2}{\PYZdq{}}\PY{p}{)}
    \end{Verbatim}
\end{tcolorbox}

\begin{Verbatim}[commandchars=\\\{\}]
    The average used space for the third case is 0.998 TB with a standard deviation
    of 0.001 TB
\end{Verbatim}

The standard deviation decreases as the block size decreases, meaning that the
used memory distribution will concentrate in a very small portion of the event
space and that the available memory will be used optimally for all hard disks.

Since the used space is obtained as a sum of random variables (with the
probability to add a file for a hard disk inversely proportional to the
number of hard disk), the sum of the memory used by single files, for
the central limit theorem, must behave as a Gaussian distribution with
mean and variance similar to the ones found before.

This proposition can be verified simulating the process for an higher
number of disks and plotting the distribution of the used space.

\begin{tcolorbox}[breakable, size=fbox, boxrule=1pt, pad at break*=1mm,colback=cellbackground, colframe=cellborder]
    \prompt{In}{incolor}{29}{\boxspacing}
    \begin{Verbatim}[commandchars=\\\{\}]
        \PY{n}{occt1} \PY{o}{=} \PY{n}{np}\PY{o}{.}\PY{n}{zeros}\PY{p}{(}\PY{l+m+mi}{2000}\PY{p}{)}
        \PY{k}{while}  \PY{p}{(}\PY{o+ow}{not} \PY{n}{np}\PY{o}{.}\PY{n}{any}\PY{p}{(}\PY{n}{occt1} \PY{o}{\PYZgt{}}\PY{o}{=} \PY{l+m+mf}{1.}\PY{p}{)}\PY{p}{)}\PY{p}{:}
        \PY{n}{occt1}\PY{p}{[}\PY{n}{np}\PY{o}{.}\PY{n}{random}\PY{o}{.}\PY{n}{randint}\PY{p}{(}\PY{l+m+mi}{0}\PY{p}{,}\PY{l+m+mi}{2000}\PY{p}{)}\PY{p}{]}\PY{o}{+}\PY{o}{=}\PY{l+m+mf}{0.01}
        \PY{n}{occt2} \PY{o}{=} \PY{n}{np}\PY{o}{.}\PY{n}{zeros}\PY{p}{(}\PY{l+m+mi}{2000}\PY{p}{)}
        \PY{k}{while}  \PY{p}{(}\PY{o+ow}{not} \PY{n}{np}\PY{o}{.}\PY{n}{any}\PY{p}{(}\PY{n}{occt2} \PY{o}{\PYZgt{}}\PY{o}{=} \PY{l+m+mf}{1.}\PY{p}{)}\PY{p}{)}\PY{p}{:}
        \PY{n}{occt2}\PY{p}{[}\PY{n}{np}\PY{o}{.}\PY{n}{random}\PY{o}{.}\PY{n}{randint}\PY{p}{(}\PY{l+m+mi}{0}\PY{p}{,}\PY{l+m+mi}{2000}\PY{p}{)}\PY{p}{]}\PY{o}{+}\PY{o}{=}\PY{l+m+mf}{0.001}
        \PY{c+c1}{\PYZsh{}\PYZsh{} The simulation with a larger number of disks for the case}
        \PY{c+c1}{\PYZsh{}\PYZsh{} with 4 MB blocks takes a lot of much time}
        \PY{c+c1}{\PYZsh{}occt3 = np.zeros(500)}
        \PY{c+c1}{\PYZsh{}while  (not np.any(occt3 \PYZgt{}= 1.)):}
        \PY{c+c1}{\PYZsh{}    occt3[np.random.randint(0,500)]+=4e\PYZhy{}6}
    \end{Verbatim}
\end{tcolorbox}

\begin{tcolorbox}[breakable, size=fbox, boxrule=1pt, pad at break*=1mm,colback=cellbackground, colframe=cellborder]
    \prompt{In}{incolor}{30}{\boxspacing}
    \begin{Verbatim}[commandchars=\\\{\}]
        \PY{n}{fig}\PY{p}{,}\PY{n}{ax} \PY{o}{=} \PY{n}{plt}\PY{o}{.}\PY{n}{subplots}\PY{p}{(}\PY{l+m+mi}{1}\PY{p}{,}\PY{l+m+mi}{1}\PY{p}{,} \PY{n}{figsize} \PY{o}{=} \PY{p}{(}\PY{l+m+mi}{7}\PY{p}{,}\PY{l+m+mi}{5}\PY{p}{)}\PY{p}{)}
        \PY{n}{ax}\PY{o}{.}\PY{n}{hist}\PY{p}{(}\PY{n}{occt1}\PY{p}{,} \PY{n}{alpha} \PY{o}{=} \PY{l+m+mf}{0.7}\PY{p}{,} \PY{n}{bins} \PY{o}{=} \PY{l+m+mi}{50}\PY{p}{,} \PY{n}{density} \PY{o}{=} \PY{k+kc}{True}\PY{p}{,} \PY{n}{label}\PY{o}{=}\PY{l+s+s2}{\PYZdq{}}\PY{l+s+s2}{10 GB block size}\PY{l+s+s2}{\PYZdq{}}\PY{p}{)}
        \PY{n}{ax}\PY{o}{.}\PY{n}{hist}\PY{p}{(}\PY{n}{occt2}\PY{p}{,}\PY{n}{alpha} \PY{o}{=} \PY{l+m+mf}{0.7}\PY{p}{,} \PY{n}{bins} \PY{o}{=} \PY{l+m+mi}{50}\PY{p}{,} \PY{n}{density} \PY{o}{=} \PY{k+kc}{True}\PY{p}{,} \PY{n}{label}\PY{o}{=}\PY{l+s+s2}{\PYZdq{}}\PY{l+s+s2}{1 GB block size}\PY{l+s+s2}{\PYZdq{}}\PY{p}{)}
        \PY{c+c1}{\PYZsh{}plt.hist(occt3,alpha = 0.7, bins =25)}
        \PY{n}{ax}\PY{o}{.}\PY{n}{set\PYZus{}xlabel}\PY{p}{(}\PY{l+s+s2}{\PYZdq{}}\PY{l+s+s2}{Used memory[TB]}\PY{l+s+s2}{\PYZdq{}}\PY{p}{)}
        \PY{n}{ax}\PY{o}{.}\PY{n}{set\PYZus{}ylabel}\PY{p}{(}\PY{l+s+s2}{\PYZdq{}}\PY{l+s+s2}{Normalized counts}\PY{l+s+s2}{\PYZdq{}}\PY{p}{)}

        \PY{n}{plt}\PY{o}{.}\PY{n}{legend}\PY{p}{(}\PY{p}{)}
        \PY{n}{plt}\PY{o}{.}\PY{n}{show}\PY{p}{(}\PY{p}{)}
    \end{Verbatim}
\end{tcolorbox}

\begin{center}
    \adjustimage{max size={0.9\linewidth}{0.9\paperheight}}{Final_assignment_files/Final_assignment_73_0.png}
\end{center}
{ \hspace*{\fill} \\}

The two gaussians behaves has expected.

\hypertarget{section}{%
    \subsection*{4}\label{section}}

\hypertarget{section}{%
    \subsubsection*{4.1.1}\label{section}}

All the operation are done on the server using the package
\texttt{requests}. The package allows to send GET and POST and so to do
all the operations required.

\begin{tcolorbox}[breakable, size=fbox, boxrule=1pt, pad at break*=1mm,colback=cellbackground, colframe=cellborder]
    \prompt{In}{incolor}{2}{\boxspacing}
    \begin{Verbatim}[commandchars=\\\{\}]
        \PY{n}{URL} \PY{o}{=} \PY{l+s+s2}{\PYZdq{}}\PY{l+s+s2}{https://pansophy.app:8443}\PY{l+s+s2}{\PYZdq{}}
    \end{Verbatim}
\end{tcolorbox}

\texttt{get} allows to download all the content of the server, while
\texttt{post} is used to append new transactions.

\begin{tcolorbox}[breakable, size=fbox, boxrule=1pt, pad at break*=1mm,colback=cellbackground, colframe=cellborder]
    \prompt{In}{incolor}{ }{\boxspacing}
    \begin{Verbatim}[commandchars=\\\{\}]
        \PY{n}{r} \PY{o}{=} \PY{n}{req}\PY{o}{.}\PY{n}{get}\PY{p}{(}\PY{n}{url} \PY{o}{=} \PY{n}{URL}\PY{p}{,} \PY{n}{verify}\PY{o}{=}\PY{k+kc}{False}\PY{p}{)}
    \end{Verbatim}
\end{tcolorbox}

A json is used to append requests to the server.

\begin{tcolorbox}[breakable, size=fbox, boxrule=1pt, pad at break*=1mm,colback=cellbackground, colframe=cellborder]
    \prompt{In}{incolor}{40}{\boxspacing}
    \begin{Verbatim}[commandchars=\\\{\}]
        \PY{n}{data} \PY{o}{=} \PY{p}{\PYZob{}}\PY{l+s+s2}{\PYZdq{}}\PY{l+s+s2}{operation}\PY{l+s+s2}{\PYZdq{}}\PY{p}{:} \PY{l+s+s2}{\PYZdq{}}\PY{l+s+s2}{merit}\PY{l+s+s2}{\PYZdq{}}\PY{p}{,}
        \PY{l+s+s2}{\PYZdq{}}\PY{l+s+s2}{team}\PY{l+s+s2}{\PYZdq{}}\PY{p}{:} \PY{l+s+s2}{\PYZdq{}}\PY{l+s+s2}{Giovanni}\PY{l+s+s2}{\PYZdq{}}\PY{p}{,}
        \PY{l+s+s2}{\PYZdq{}}\PY{l+s+s2}{coin}\PY{l+s+s2}{\PYZdq{}}\PY{p}{:} \PY{l+m+mi}{1000}\PY{p}{,}
        \PY{l+s+s2}{\PYZdq{}}\PY{l+s+s2}{stealfrom}\PY{l+s+s2}{\PYZdq{}}\PY{p}{:} \PY{l+s+s2}{\PYZdq{}}\PY{l+s+s2}{genesis}\PY{l+s+s2}{\PYZdq{}}\PY{p}{\PYZcb{}}
    \end{Verbatim}
\end{tcolorbox}

\begin{tcolorbox}[breakable, size=fbox, boxrule=1pt, pad at break*=1mm,colback=cellbackground, colframe=cellborder]
    \prompt{In}{incolor}{ }{\boxspacing}
    \begin{Verbatim}[commandchars=\\\{\}]
        \PY{n}{resp1} \PY{o}{=} \PY{n}{req}\PY{o}{.}\PY{n}{post}\PY{p}{(}\PY{n}{url} \PY{o}{=} \PY{n}{URL}\PY{p}{,} \PY{n}{json} \PY{o}{=} \PY{n}{data}\PY{p}{,} \PY{n}{verify}\PY{o}{=}\PY{k+kc}{False}\PY{p}{)}
    \end{Verbatim}
\end{tcolorbox}

Another json is needed in order to claim the amount required. This one
must be ``posted'' after 10s to pass the proof of time.

\begin{tcolorbox}[breakable, size=fbox, boxrule=1pt, pad at break*=1mm,colback=cellbackground, colframe=cellborder]
    \prompt{In}{incolor}{42}{\boxspacing}
    \begin{Verbatim}[commandchars=\\\{\}]
        \PY{n}{claim} \PY{o}{=} \PY{p}{\PYZob{}}
        \PY{l+s+s2}{\PYZdq{}}\PY{l+s+s2}{operation}\PY{l+s+s2}{\PYZdq{}}\PY{p}{:} \PY{l+s+s2}{\PYZdq{}}\PY{l+s+s2}{claim}\PY{l+s+s2}{\PYZdq{}}\PY{p}{,}
        \PY{l+s+s2}{\PYZdq{}}\PY{l+s+s2}{team}\PY{l+s+s2}{\PYZdq{}}\PY{p}{:} \PY{l+s+s2}{\PYZdq{}}\PY{l+s+s2}{Giovanni}\PY{l+s+s2}{\PYZdq{}}
        \PY{p}{\PYZcb{}}
    \end{Verbatim}
\end{tcolorbox}

\begin{tcolorbox}[breakable, size=fbox, boxrule=1pt, pad at break*=1mm,colback=cellbackground, colframe=cellborder]
    \prompt{In}{incolor}{ }{\boxspacing}
    \begin{Verbatim}[commandchars=\\\{\}]
        \PY{n}{resp2} \PY{o}{=} \PY{n}{req}\PY{o}{.}\PY{n}{post}\PY{p}{(}\PY{n}{url} \PY{o}{=} \PY{n}{URL}\PY{p}{,} \PY{n}{json} \PY{o}{=} \PY{n}{claim}\PY{p}{,} \PY{n}{verify}\PY{o}{=}\PY{k+kc}{False}\PY{p}{)}
    \end{Verbatim}
\end{tcolorbox}

\hypertarget{section}{%
    \subsubsection*{4.1.2}\label{section}}

The maximum number of transactions that can be done in a day, given that
only one transaction can be done at once and that the proof of time is
10s is simply calculated as

\[\frac{24 \cdot 60 \cdot 60 s}{10 s}= 8640\]

\hypertarget{section}{%
    \subsubsection*{4.2/4.2.1}\label{section}}

\begin{tcolorbox}[breakable, size=fbox, boxrule=1pt, pad at break*=1mm,colback=cellbackground, colframe=cellborder]
    \prompt{In}{incolor}{56}{\boxspacing}
    \begin{Verbatim}[commandchars=\\\{\}]
        \PY{k}{def} \PY{n+nf}{calculate\PYZus{}hash}\PY{p}{(}\PY{n+nb+bp}{self}\PY{p}{)}\PY{p}{:}
        \PY{n}{block\PYZus{}of\PYZus{}string} \PY{o}{=} \PY{l+s+s2}{\PYZdq{}}\PY{l+s+si}{\PYZob{}\PYZcb{}}\PY{l+s+si}{\PYZob{}\PYZcb{}}\PY{l+s+si}{\PYZob{}\PYZcb{}}\PY{l+s+si}{\PYZob{}\PYZcb{}}\PY{l+s+si}{\PYZob{}\PYZcb{}}\PY{l+s+s2}{\PYZdq{}}\PY{o}{.}\PY{n}{format}\PY{p}{(} \PY{n+nb+bp}{self}\PY{o}{.}\PY{n}{index}\PY{p}{,}
        \PY{n+nb+bp}{self}\PY{o}{.}\PY{n}{team}\PY{p}{,}
        \PY{n+nb+bp}{self}\PY{o}{.}\PY{n}{prev\PYZus{}hash}\PY{p}{,}
        \PY{n+nb+bp}{self}\PY{o}{.}\PY{n}{coins}\PY{p}{,}
        \PY{n+nb+bp}{self}\PY{o}{.}\PY{n}{timestamp}\PY{p}{)}
        \PY{n+nb+bp}{self}\PY{o}{.}\PY{n}{my\PYZus{}hash} \PY{o}{=} \PY{n}{hashlib}\PY{o}{.}\PY{n}{sha256}\PY{p}{(}\PY{n}{block\PYZus{}of\PYZus{}string}\PY{o}{.}\PY{n}{encode}\PY{p}{(}\PY{p}{)}\PY{p}{)}\PY{o}{.}\PY{n}{hexdigest}\PY{p}{(}\PY{p}{)}
        \PY{k}{return} \PY{n+nb+bp}{self}\PY{o}{.}\PY{n}{my\PYZus{}hash}
    \end{Verbatim}
\end{tcolorbox}

This function works as an hashing function that depends on the structure
and the timestamp of a transaction, and also on the previous transaction
hash.

The used hashing algorithm is SHA256 that generates a 32 bytes hash for
each given set of values encoded in the variable
\texttt{block\_of\_string}.

Using this kind of hashing algorithm allows to identify univocally each
block inside the blockchain and make the new hash depend on the hash of
the previous block. The security implications will be treated below.

\hypertarget{section}{%
    \subsubsection*{4.2.2}\label{section}}

The validation of the blockchain's content is done by checking that all
the hashes actually corresponds to the blocks' contents. This must be
done by hashing all the blocks of the blockchain, since each one depends
on the previous one. A small change of a block has so a consequence
easily detectable, given that the SHA256 algorithm produces a difference
hash if the starting string is different.

A variation on the blockchain can be found easily since a change of one
of the old blocks implies changes in all the next hashes, given that
each hash depends on the previous one. The existence of proof of time
also allows to keep the system more secure, since a full subsitution of
blocks would result in a large execution time before all the elements of
the blockchain are substituted with the correct ones, in order to have
consistecy between the blocks' hash.

\hypertarget{section}{%
    \subsubsection*{4.2.3}\label{section}}

POST is used to append requests to the blockchain and so the execution
time dependence is only of O(1), since it does not have to go trough all
the blocks of the chain in order to ``post'' a new one.

GET instead works by downloading all the chain content and so his
execution time dipendence must be of O(N), with N as the current number
of transactions.

The GET dependence on N can be one of the major problems in scalability
if the chain has a large size.

\begin{tcolorbox}[breakable, size=fbox, boxrule=1pt, pad at break*=1mm,colback=cellbackground, colframe=cellborder]
    \prompt{In}{incolor}{30}{\boxspacing}
    \begin{Verbatim}[commandchars=\\\{\}]
        \PY{n}{fig}\PY{p}{,} \PY{n}{ax} \PY{o}{=} \PY{n}{plt}\PY{o}{.}\PY{n}{subplots}\PY{p}{(}\PY{l+m+mi}{1}\PY{p}{,}\PY{l+m+mi}{1}\PY{p}{,}\PY{n}{figsize} \PY{o}{=} \PY{p}{(}\PY{l+m+mi}{10}\PY{p}{,}\PY{l+m+mi}{6}\PY{p}{)}\PY{p}{)}
        \PY{n}{plt}\PY{o}{.}\PY{n}{plot}\PY{p}{(}\PY{n}{np}\PY{o}{.}\PY{n}{linspace}\PY{p}{(}\PY{l+m+mi}{0}\PY{p}{,}\PY{l+m+mi}{100}\PY{p}{,}\PY{l+m+mi}{1000}\PY{p}{)}\PY{p}{,}\PY{n}{np}\PY{o}{.}\PY{n}{linspace}\PY{p}{(}\PY{l+m+mi}{0}\PY{p}{,}\PY{l+m+mi}{100}\PY{p}{,}\PY{l+m+mi}{1000}\PY{p}{)}\PY{p}{,} \PY{n}{label} \PY{o}{=} \PY{l+s+s2}{\PYZdq{}}\PY{l+s+s2}{GET function trend}\PY{l+s+s2}{\PYZdq{}}\PY{p}{)}
        \PY{n}{plt}\PY{o}{.}\PY{n}{plot}\PY{p}{(}\PY{n}{np}\PY{o}{.}\PY{n}{linspace}\PY{p}{(}\PY{l+m+mi}{0}\PY{p}{,}\PY{l+m+mi}{100}\PY{p}{,}\PY{l+m+mi}{1000}\PY{p}{)}\PY{p}{,}\PY{p}{[}\PY{l+m+mi}{30}\PY{p}{]}\PY{o}{*}\PY{l+m+mi}{1000}\PY{p}{,} \PY{n}{label} \PY{o}{=} \PY{l+s+s2}{\PYZdq{}}\PY{l+s+s2}{POST function trend}\PY{l+s+s2}{\PYZdq{}}\PY{p}{)}
        \PY{n}{ax}\PY{o}{.}\PY{n}{set\PYZus{}ylabel}\PY{p}{(}\PY{l+s+s2}{\PYZdq{}}\PY{l+s+s2}{Execution time}\PY{l+s+s2}{\PYZdq{}}\PY{p}{)}
        \PY{n}{ax}\PY{o}{.}\PY{n}{set\PYZus{}xlabel}\PY{p}{(}\PY{l+s+s2}{\PYZdq{}}\PY{l+s+s2}{Number of transactions}\PY{l+s+s2}{\PYZdq{}}\PY{p}{)}
        \PY{n}{ax}\PY{o}{.}\PY{n}{set\PYZus{}xticklabels}\PY{p}{(}\PY{p}{[}\PY{p}{]}\PY{p}{)}
        \PY{n}{ax}\PY{o}{.}\PY{n}{set\PYZus{}yticklabels}\PY{p}{(}\PY{p}{[}\PY{p}{]}\PY{p}{)}
        \PY{n}{plt}\PY{o}{.}\PY{n}{legend}\PY{p}{(}\PY{p}{)}
        \PY{n}{plt}\PY{o}{.}\PY{n}{show}\PY{p}{(}\PY{p}{)}
    \end{Verbatim}
\end{tcolorbox}

\begin{center}
    \adjustimage{max size={0.9\linewidth}{0.9\paperheight}}{Final_assignment_files/Final_assignment_95_0.png}
\end{center}
{ \hspace*{\fill} \\}

\hypertarget{section}{%
    \subsubsection*{4.2.4}\label{section}}

The account balances can be restored simply by starting from the inital
state and redistributing the money according to the sequence of
transactions.

\hypertarget{section}{%
    \subsubsection*{4.2.5}\label{section}}

The main advantage of REST APIs is the simplicity, the structure based
on GET and POST functions that allows a complete communication with the
server. Also the use of some communication standards, such as
\texttt{JSON}, allows to make the exchange of information even simpler.
In particular \texttt{JSON} structure is helpful if the information set
to send has an hierarchical shape, since it allows to store it in a
plain structure.

The main disadvantage of REST APIs is the stateless configuration, so if
the server crashes all the informations about the state are lost as was
discussed in exercise 4.2.4, and the last state must be rebuilt from the
starting one.


% Add a bibliography block to the postdoc



\end{document}
